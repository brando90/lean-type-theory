\section*{Abstract}
% \addcontentsline{toc}{chapter}{abstract}

This thesis is a presentation of dependent type theory with inductive types, a hierarchy of universes, with an impredicative universe of propositions, proof irrelevance, and subsingleton elimination, along with axioms for propositional extensionality, quotient types, and the axiom of choice. This theory is notable for being the axiomatic framework of the Lean theorem prover. The axiom system is given here in complete detail, including ``optional'' features of the type system such as $\mathsf{let}$ binders and definitions. We provide a reduction of the theory to a finitely axiomatized fragment utilizing a fixed set of inductive types (the $\W$ -type plus a few others), to ease the study of this framework.

The metatheory of this theory (which we will call Lean) is studied. In particular, we prove unique typing of the definitional equality, and use this to construct the expected set-theoretic model, from which we derive consistency of Lean relative to ZFC with $n$ inaccessible cardinals for all $n<\omega$ (a relatively weak large cardinal assumption). As Lean supports models of ZFC with $n$ inaccessible cardinals, this is optimal.

We also show a number of negative results, where the theory is less nice than we would like. In particular, type checking is undecidable, and the type checking as implemented by the Lean theorem prover is a decidable non-transitive underapproximation of the typing judgment. Non-transitivity also leads to lack of subject reduction, and the reduction relation does not satisfy the Church-Rosser property, so reduction to a normal form does not produce a decision procedure for definitional equality. However, a modified reduction relation allows us to restore the Church-Rosser property at the expense of guaranteed termination, so that unique typing is shown to hold.
