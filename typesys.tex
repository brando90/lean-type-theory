\section{Properties of the type system}
A theorem we would like to have of Lean's type system is that it is consistent, and sound with respect to some semantics in a well understood axiom system such as ZFC. Moreover, we want to relate this to Lean's actual typechecker, in the sense that anything Lean verifies as type-correct will be derivable in this axiom system and hence Lean will not certify a contradiction. But first we must understand some aspects of the type system itself, before relating it to other systems.

It is important to note that \emph{Lean's typechecker is not complete.} Obviously Lean can fail on correct theorems due to, say, running out of resources, but the ``algorithmic equality'' relation does not validate all definitional equalities. In fact, we can show that definitional equality as defined here is undecidable.

\subsection{Undecidability of definitional equality}\label{undecidable}
Recall the type $\acc$ from section \ref{large_elim}:
$$\acc_{<}:=\mu A:\alpha\to\P.\ (\intro:\forall x:\alpha.\;(\forall y:\alpha.\;y<x\to A\;y)\to A\;x)$$
(We are fixing a type $\alpha$ and a relation ${<}:\alpha\to\alpha\to\P$ here.) Informally, we would read this as: ``$x$ is $<$-accessible if for all $y<x$, $y$ is $<$-accessible''. Accessibility is then inductively generated by this clause. If every $x:\alpha$ is accessible, then $<$ is a well-founded relation. One interesting fact about $\acc$ is that we can project out the argument given a proof of $\acc\;x$:
\begin{align*}
\mathsf{inv}_x&:\acc\;x\to\forall y:\alpha.\;y<x\to\acc\;y\\
\mathsf{inv}_x&:=\lambda a:\acc\;x.\;\lambda y:\alpha.\;\rec_{\acc}\;(\lambda z.\;y<z\to\acc\;y)\\
&(\lambda z.\;\lambda h:(\forall w.\;w<z\to\acc\;w).\;\lambda \_.\;h\;y)\;x\;a
\end{align*}
Note that the output type of $\mathsf{inv}_x$ is the same as the argument to $\intro\;x$. Thus, we have
$$a\equiv\intro_\acc\;x\;(\mathsf{inv}_x\;a)$$
by proof irrelevance.

Why does this matter? Normally, any proof of $\acc\;x$ could only be unfolded finitely many times by the very nature of inductive proofs, but if we are in an inconsistent context, it is possible to get a proof of wellfoundedness which isn't actually wellfounded, and we can end up unfolding it forever.

To show how to get undecidability from this, suppose $P:\N\to\bf 2$ is a decidable predicate, such as $P\;n:=\;$``Turing machine $M$ runs for at least $n$ steps without halting''. It is not difficult to show that $P\;n$ is decidable but $\forall n.\;P\;n$ is not. Let $>$ be the standard greater-than function on $\N$ (which is not well-founded). We define a function $f:\forall n.\;\acc_{>}\;n\to\bf 1$ as follows:
\begin{align*}
f&:=\rec_{\acc}\;(\lambda\_.\;{\bf 1})\;(\lambda n\;\_\;(g:\forall y.\;y>x\to{\bf 1}).\\
&\qquad\mathsf{if}\;P\;n\;\mathsf{then}\;g\;(n+1)\;(p\;n)\;\mathsf{else}\;()
\end{align*}
where $p\;n$ is a proof of $n<n+1$. Of course this whole function is trivial since the precondition $\acc_{>}n$ is impossible, but definitional equality works in all contexts, including inconsistent ones. This function evaluates as:
$$f\;n\;(\intro_\acc\;n\;h)\rightsquigarrow^*\mathsf{if}\;P\;n\;\mathsf{then}\;f\;(n+1)\;(h\;(n+1)\;(p\;n))\;\mathsf{else}\;()$$
and the \textsf{if} statement evaluates to the left or right branch depending on whether $P\;n\rightsquigarrow^*\mathsf{tt}$ or $P\;n\rightsquigarrow^*\mathsf{ff}$. Now, this is all true of the reduction relation $\rightsquigarrow$, but if we bring in the full power of definitional equivalence we have the ability to work up from a single proof $a:\acc_{>}\;0$:
\begin{align*}
f\;0\;a&\equiv f\;0\;(\intro_\acc\;0\;(\mathsf{inv}_0\;a))\\
&\equiv f\;1\;(\mathsf{inv}_0\;a\;1\;(p\;0))\\
&\equiv f\;1\;(\intro_\acc\;1\;(\mathsf{inv}_1\;(\mathsf{inv}_0\;a\;1\;(p\;0)))\\
&\equiv f\;2\;(\mathsf{inv}_1\;(\mathsf{inv}_0\;a\;1\;(p\;0))\;2\;(p\;1))\\
&\equiv\dots
\end{align*}
where we have shown the case where $P\;0$ and $P\;1$ both evaluate to true. If any $P\;n$ evaluates to false, then we will eventually get an equivalence to $()$, but if $P\;n$ is always true, then $f$ will never reduce to $()$ -- every term definitionally equal to $f\;0\;a$ will contain a subterm def.eq. to $f$. So $a:\acc_{>}\;0\vdash f\;0\;a\equiv()$ holds if and only if $\forall n.\;P\;n$, and hence $\equiv$ is undecidable.

\subsubsection{Algorithmic equality is not transitive}
From the results of the previous section, given that algorithmic equality is implemented by Lean, and hence is obviously decidable, they cannot be equal as relations, so there is some rule of definitional equality that is not respected by algorithmic equality. In the above example, we can typecheck the various parts of the equality chain to see that $\Leftrightarrow$ is not transitive:
\begin{align*}
f\;0\;a&\Leftrightarrow f\;0\;(\intro_\acc\;0\;(\mathsf{inv}_0\;a))\\
&\Leftrightarrow f\;1\;(\mathsf{inv}_0\;a\;1\;(p\;0))\\
&\mbox{but}\\
f\;0\;a&\not\Leftrightarrow f\;1\;(\mathsf{inv}_0\;a\;1\;(p\;0)).
\end{align*}
We can think of the middle step $f\;0\;(\intro_\acc\;0\;(\mathsf{inv}_0\;a))$ as a ``creative'' step, where we pick one of the many possible terms of type $\acc_{>}\;0$ which happens to reduce in the right way. But since the expression $f\;0\;a$ is a normal form, we don't attempt to reduce it, and indeed if we did we would have nontermination problems (since reduction here only makes the term larger).

Note that the fact that we are in an inconsistent context doesn't matter for this: we could have used $a:\acc_{<}\;1$ with the same result.

This instance of non-transitivity can be traced back to the usage of a K-like eliminator via $\acc$. There is another, less known source of non-transitivity: quotients of propositions. While this is not a particularly useful operation, since any proposition is already a subsingleton, so a quotient will not do anything, they can technically be formed, and $\lift$ acts like a K-like eliminator in this case. So for example, if $p:\P$, $R:p\to p\to\P$, $\alpha:\U_1$, $f:p\to\alpha$, $H:\forall x\;y.\;r\;x\;y\to f\;x= f\;y$, $q:p/R$ and $h:p$, then:
\begin{align*}
\lift_R\;\alpha\;f\;H\;q&\Leftrightarrow \lift_R\;\alpha\;f\;H\;(\mk_R\;h)\Leftrightarrow f\;h\\
&\mbox{but}\\
\lift_R\;\alpha\;f\;H\;q&\not\Leftrightarrow f\;h.
\end{align*}

\subsubsection{Failure of subject reduction}
While the type system given here actually satisfies subject reduction (which is to say, if $\Gamma\vdash e:\alpha$ and $e\rightsquigarrow e'$ (or $\Gamma\vdash e\Leftrightarrow e'$, or $\Gamma\vdash e\equiv e'$), then $\Gamma\vdash e':\alpha$), this is because we use the $\equiv$ relation in the conversion rule $\Gamma\vdash e:\alpha$, $\Gamma\vdash \alpha\equiv\beta$ implies $\Gamma\vdash e:\beta$. If we used algorithmic equality instead, to get a variant typing judgment $\Gamma\Vdash e:\alpha$ closer to what one would expect of the Lean typechecker, we find failure of subject reduction, directly from failure of transitivity. If $\Gamma\vdash\alpha\Leftrightarrow\beta$, $\Gamma\vdash\beta\Leftrightarrow\gamma$, $\Gamma\vdash\alpha\not\Leftrightarrow\gamma$, and $\Gamma\Vdash e:\gamma$, then:
\begin{itemize}
\item $\Gamma\Vdash \mbox{id}_\beta\;e:\beta$ because the application forces checking $\Gamma\vdash\beta\Leftrightarrow\gamma$.
\item $\Gamma\Vdash \mbox{id}_\alpha\;(\mbox{id}_\beta\;e):\alpha$ since the application forces checking $\Gamma\vdash\alpha\Leftrightarrow\beta$.
\item But $\Gamma\not\Vdash \mbox{id}_\alpha\;e:\alpha$ because this requires $\Gamma\vdash\alpha\Leftrightarrow\gamma$ which is false.
\end{itemize}
Since we obviously have $\mbox{id}_\beta\;e\rightsquigarrow e$ by the $\beta$ and $\delta$ rules, this is a counterexample to subject reduction.
\subsection{Regularity}
These lemmas are essentially trivial inductions and are true by virtue of the way we set up the type system, so they are recorded here simply to keep track of the invariants.

\begin{lemma}[Regularity]\label{reg}
\begin{enumerate}
\item If $\Gamma\vdash e:\alpha$, then $\vdash\Gamma\;\mathsf{ok}$.
\item If $\Gamma\vdash e:\alpha$, then $FV(e)\cup FV(\alpha)\subseteq\Gamma$.
\item If $\Gamma\vdash\alpha\type$, then $\Gamma\vdash\alpha:\U_\ell$ for some $\ell$.
\item If $\Gamma\vdash e:\alpha$, then $\Gamma\vdash\alpha\type$.
\item\label{defeq_reg2} If $\Gamma\vdash e\equiv e'$, then there exists $\alpha,\alpha'$ such that $\Gamma\vdash e:\alpha$ and $\Gamma\vdash e':\alpha'$.
\item If $\Gamma\vdash e:\alpha$ and $e\rightsquigarrow e'$, then $\Gamma\vdash e\equiv e'$.
\item If $\Gamma\vdash e:\alpha$ and $e\downarrow k$, then $\Gamma\vdash e\equiv k$.
\item\label{alg_defn} If $\Gamma\vdash e:\alpha$ and $\Gamma\vdash e':\alpha$, and $\Gamma\vdash e\Leftrightarrow e'$, then $\Gamma\vdash e\equiv e'$.
\item If $\Gamma;t:F\vdash K\spec$, then $\Gamma\vdash F\type$ (and more precisely, $F=\forall x::\alpha.\;\U_\ell$ for some $\alpha,\ell$).
\item If $\Gamma;t:F\vdash K\spec$ and $(c:\alpha)\in K$, then $\Gamma;t:F\vdash\alpha\ctor$.
\item If $\Gamma;t:F\vdash \alpha\ctor$, then $\Gamma,t:F\vdash\alpha\type$.
\end{enumerate}
\end{lemma}
\begin{proof}
By induction on the respective judgments (all of the parts may be proven separately).
\end{proof}

\begin{lemma}[Weakening]\label{weak}
\begin{enumerate}
\item If $\Gamma\vdash e:\alpha$ and $\vdash\Gamma,\Delta\ok$, then $\Gamma,\Delta\vdash e:\alpha$.
\item If $\Gamma\vdash e\equiv e'$ and $\vdash\Gamma,\Delta\ok$, then $\Gamma,\Delta\vdash e\equiv e'$.
\item If $\Gamma,\Delta\vdash e:\alpha$ and $FV(e)\subseteq\Gamma$, then $\Gamma\vdash e:\alpha$.
\item If $\Gamma,\Delta\vdash e\equiv e'$ and $FV(e)\cup FV(e')\subseteq\Gamma$, then $\Gamma\vdash e\equiv e'$.
\item $\Gamma\vdash e:\alpha$ implies $\Gamma\vdash' e:\alpha$, and $\Gamma\vdash e\equiv e'$ implies $\Gamma\vdash' e\equiv e'$, where the modified judgment $\vdash'$ eliminates the weakening rules and replaces the variable and universe rules with
$$\frac{(x:\alpha)\in\Gamma}{\Gamma\vdash x:\alpha}\qquad\frac{}{\Gamma\vdash \U_\ell:\U_{S\ell}}\qquad\frac{\ell\equiv\ell'}{\Gamma\vdash \U_\ell\equiv\U_{\ell'}}$$
\end{enumerate}
\end{lemma}
\begin{proof}
(1,2) and (3,4) are each proven by mutual induction on the first hypothesis. For (5), since weakening is provable for the judgment $\vdash'$ it follows that all rules of $\vdash$ are provable in $\vdash'$.
\end{proof}

\begin{lemma}[Properties of substitution]\label{subst}
\begin{enumerate}
\item If $\Gamma,x:\alpha\vdash e_1\equiv e_1'$ and $\Gamma\vdash e_2:\alpha$, then $\Gamma\vdash e_1[e_2/x]\equiv e_1'[e_2/x]$.
\item\label{subst_ty} If $\Gamma,x:\alpha\vdash e_1:\beta$ and $\Gamma\vdash e_2:\alpha$, then $\Gamma\vdash e_1[e_2/x]:\beta[e_2/x]$.
\item If $\Gamma,x:\alpha\vdash e_1:\beta$ and $\Gamma\vdash e_2\equiv e_2':\alpha$, then $\Gamma\vdash e_1[e_2/x]\equiv e_1[e_2'/x]$.
\end{enumerate}
\end{lemma}
\begin{proof} (1) and (2) must be proven simultaneously by induction on the first hypotheses. All cases are straightforward. In the proof irrelevance case, we know $\Gamma,x:\alpha\vdash e_1:p$ and $\Gamma,x:\alpha\vdash e_1':p$ for some $p$ with $\Gamma,x:\alpha\vdash p:\P$. By the induction hypothesis, $\Gamma\vdash e_1[e_2/x]:p[e_2/x]$ and $\Gamma\vdash e_1'[e_2/x]:p[e_2/x]$ and $\Gamma\vdash p[e_2/x]:\P[e_2/x]$; but $\P[e_2/x]=\P$ so proof irrelevance applies to show $\Gamma\vdash e_1[e_2/x]=e_1'[e_2/x]$.

(3) is proven by induction on the structure of $e_1$ and applying compatibility lemmas in each case.
\end{proof}

With this theorem we can upgrade lemma \ref{reg}(\ref{defeq_reg2}) to:
\begin{lemma}[Regularity continued]
\begin{enumerate}
\item If $\Gamma\vdash e\equiv e'$, then there exists $\alpha$ such that $\Gamma\vdash e\equiv e':\alpha$.
\end{enumerate}
\end{lemma}
\begin{proof}
Straightforward induction on the derivation of $\Gamma\vdash e\equiv e'$. We need lemma \ref{subst}(\ref{subst_ty}) to typecheck both sides of the $\beta$ rule. Note that the induction hypothesis is not strong enough for the application rule, except that we explicitly require that both sides have agreeing types in this case.
\end{proof}

\subsection{Minimal derivations}
Let the notation $\Gamma\vdash_0e:\alpha$ mean that $\Gamma\vdash e:\alpha$, and there is no derivation of  $\Gamma\vdash' e:\alpha'$ that has fewer steps than some derivation of $\Gamma\vdash' e:\alpha$ (where the number of steps of a derivation is the sum of the number of steps of the hypotheses to the rule plus one), where the $\vdash'$ derivation is the weakening-free judgment defined in lemma \ref{weak}. (Alternatively, we could count steps in a $\vdash$ derivation but ignore weakening steps.) The step counting does not include embedded derivations of $\Gamma\vdash e\equiv e'$, it only counts the typing steps.

\begin{lemma}[Properties of $\vdash_0$]
\begin{enumerate}
\item If $\Gamma\vdash e:\beta$, then $\Gamma\vdash_0e:\alpha$ for some $\alpha$.
\item The final rule in a derivation of $\Gamma\vdash_0e:\alpha$ cannot be the conversion rule.
\item If $\Gamma\vdash e:\alpha$, then either $\Gamma\vdash_0 e:\alpha$, or there is a minimal derivation whose last step is a conversion $\Gamma\vdash\beta\equiv\alpha$ where $\Gamma\vdash_0e:\beta$.
\item If $\Gamma,\Delta\vdash_0e:\alpha$ and $FV(e)\subseteq\Gamma$, then $\Gamma\vdash_0e:\alpha$.
\item If $\Gamma\vdash_0x:\alpha$, then $(x:\alpha)\in\Gamma$.
\item If $\Gamma\vdash_0\U_\ell:\alpha$, then $\alpha=\U_{S\ell}$.
\item If $\Gamma\vdash_0 e_1\;e_2:\gamma$, then $\gamma=\beta[e_2/x]$, $\Gamma\vdash e_1:\forall x:\alpha.\;\beta$, and $\Gamma\vdash e_2:\alpha$ for some $\alpha,\beta,x$.
\item If $\Gamma\vdash_0 \lambda x:\alpha.\;e:\gamma$, then $\gamma=\forall x:\alpha.\;\beta$ and $\Gamma,x:\alpha\vdash_0 e:\beta$ for some $\alpha,\beta,x$.
\item If $\Gamma\vdash_0 \forall x:\alpha.\;e:\gamma$, then $\gamma=\U_{\imax(\ell_1,\ell_2)}$, $\Gamma\vdash \alpha:\U_{\ell_1}$, and $\Gamma,x:\alpha\vdash \beta:\U_{\ell_2}$ for some $\ell_1,\ell_2,\beta,x$.
\item If $\Gamma\vdash_0 \elet{x:\alpha:=e'}{e}:\beta$, then $\Gamma\vdash e':\alpha$ and $\Gamma\vdash_0 e[e'/x]:\beta$.
\item If $\Gamma\vdash_0 c_{\bar\ell}:\alpha$, then $\alpha=\tau_{\bar\ell}(c)$ (this includes inductive types and defined and axiomatic constants).
\end{enumerate}
\end{lemma}
\begin{proof}\ \\[-1em]
\begin{enumerate}
\item Trivial from the definition.
\item If the last rule was the conversion rule, then the first hypothesis $\Gamma\vdash e:\alpha'$ would have a shorter proof.
\item This follows from the fact that a proof can be weakened by adding $\Delta$ in every step of the proof without increasing the total length, so that a shorter proof of $\Gamma\vdash e:\alpha'$ could be weakened to a short proof of $\Gamma,\Delta\vdash e:\alpha'$.
\item[4-10.] By inversion, we obtain the conclusion in each case without the $\vdash_0$ on the inverted hypotheses. In (7), we know that the derivation $\Gamma,x:\alpha\vdash e:\beta$ is minimal because if $\Gamma,x:\alpha\vdash e:\beta'$ is shorter then we can derive $\Gamma\vdash \lambda x:\alpha.\;e:\forall x:\alpha.\;\beta'$ with a shorter proof, and in (9), if $\Gamma\vdash \elet{x:\alpha:=e'}{e}:\beta'$ is a shorter proof, then $\Gamma\vdash_0 \elet{x:\alpha:=e'}{e}:\beta'$ also has a shorter proof.
\end{enumerate}
\end{proof}

\begin{lemma}[Unique typing]
If $\Gamma\vdash e:\alpha$, then:
\begin{itemize}
\item Either $\Gamma\vdash_0 e:\alpha$, or there is a minimal derivation whose last step is a conversion $\Gamma\vdash\beta\equiv\alpha$ where $\Gamma\vdash_0e:\beta$;
\item If $\Gamma\vdash_0 e:\alpha_1$ and $\Gamma\vdash_0 e:\alpha_2$, then $\Gamma\vdash\alpha_1\equiv\alpha_2$.
\item If $\Gamma\vdash_0 e:\forall x:\beta.\;\alpha_1$ and $\Gamma\vdash_0 e:\forall x:\beta.\;\alpha_2$, then $\Gamma,x:\beta\vdash\alpha_1\equiv\alpha_2$.
\end{itemize}
\end{lemma}
\begin{proof}
By induction on the length of a minimal proof of $\Gamma\vdash e:\alpha$. Note that the two parts of the inductive hypothesis imply that if $\Gamma\vdash_0 e:\alpha'$ then $\Gamma\vdash\alpha'\equiv\alpha$.
\begin{itemize}
\item If the last rule is a base case rule (variable, universe, constant), then it has length 1 and so is minimal; in this case by the inversion lemma the latter part is trivially satisfied.
\item If the last rule is the conversion rule, from hypotheses $\Gamma\vdash e:\beta$ and $\Gamma\vdash\beta\equiv\alpha$, then the inductive hypothesis applies:
\begin{itemize}
\item If $\Gamma\vdash_0 e:\beta$, then the last step is a conversion from $\Gamma\vdash_0 e:\beta$.
\item If there is a minimal derivation of $\Gamma\vdash e:\beta$ via $\Gamma\vdash_0e:\gamma$, $\Gamma\vdash\gamma\equiv\beta$, then we may join the definitional equalities by transitivity to obtain a proof $\Gamma\vdash e:\alpha$ that has the same length as the proof of $\Gamma\vdash e:\beta$ (recall that proof steps for $\Gamma\vdash e\equiv e'$ don't count toward the step count).
\end{itemize}
The second clause is immediate from the IH.
\item If the last step is the application rule, say $\dfrac{\Gamma\vdash e_1:\forall x:\alpha.\;\beta\quad \Gamma\vdash e_2:\alpha}{\Gamma\vdash e_1\;e_2:\beta[e_2/x]}$,
consider also a proof $\dfrac{\Gamma\vdash e_1:\forall x:\alpha'.\;\beta'\quad\Gamma\vdash e_2:\alpha'}{\Gamma\vdash_0 e_1\;e_2:\beta'[e_2/x]}$ (by the inversion lemma). Since all four hypotheses are shorter than the original proof, the inductive hypothesis applies in each case.

We may assume WLOG that $\Gamma\vdash_0 e_2:\alpha'$, because if not it is obtained from conversion on some $\alpha''$, and then since $\forall x:\alpha''.\;\beta'\equiv\forall x:\alpha'.\;\beta'$ we may eliminate the conversion on the right and possibly introduce a conversion on the left for no net increase in the length of the overall proof.

Similarly, since by the inductive hypothesis on $\Gamma\vdash e_2:\alpha$ now $\alpha\equiv\alpha'$, we may arrange it so that the final step of the proof is $\dfrac{\Gamma\vdash e_1:\forall x:\alpha'.\;\beta\quad \Gamma\vdash_0 e_2:\alpha'}{\Gamma\vdash e_1\;e_2:\beta[e_2/x]}$.
By the inductive hypothesis on $\Gamma\vdash e_1:\forall x:\alpha'.\;\beta'$,
\begin{itemize}
\item If both derivations are 
\end{itemize} have $\gamma'\equiv\forall x:\alpha.\;\beta$ and $\alpha'\equiv\alpha$ such that $\Gamma\vdash_0 e_1:\gamma'$ and $\Gamma\vdash_0 e_2:\alpha'$.
\end{itemize}
UNFINISHED
\end{proof}
Now let $\vdash_0\Gamma\ok$ mean that all the types in the context are minimally derived, that is:
$$\frac{}{\vdash_0\cdot\ok}\qquad\frac{\vdash_0\Gamma\ok\quad\Gamma\vdash_0\alpha:\U_\ell}{}\qquad$$
Note that the final rule in such a minimal derivation cannot be the conversion rule, because it has a hypothesis that is a shorter typing of the same expression. Also, if the last rule is weakening, so $\Gamma,x:\beta\vdash_0e:\alpha$ derives from $\Gamma\vdash e:\alpha$, then also $\Gamma\vdash_0e:\alpha$ because any shorter derivation of $\Gamma\vdash e:\alpha'$ could be similarly weakened to a proof of $\Gamma,x:\beta\vdash e:\alpha'$.

UNFINISHED


\subsection{Sort inference}
Our first tool for ensuring that the typing and definitional equality relations are not trivial is to construct a parallel typing and definitional equality judgment that keeps track of a lot more information than the ``official'' one. This will make inductive proofs easier, and then we only have to show that any typing derivation can be ``upgraded'' in this way.
$$\Delta::=\cdot\mid\Delta,x:e:\U_\ell\qquad
\boxed{\Delta\vdash e:\alpha:\U_\ell}$$
$$\frac{\Delta\vdash\alpha:\U_\ell\type\quad\Delta\vdash e:\beta:\U_{\ell'}}{\Delta,x:\alpha:\U_\ell\vdash e:\beta:\U_{\ell'}}\qquad
\frac{\Delta\vdash\alpha:\U_\ell\type}{\Delta,x:\alpha:\U_\ell\vdash x:\alpha:\U_\ell}\qquad
\frac{}{\vdash\U_\ell:\U_{S\ell}\type}$$
$$\frac{\begin{matrix}
\Delta,x:\alpha:\U_{\ell_1}\vdash\beta:\U_{\ell_2}\type\\
\Delta\vdash e_1:\forall x:\alpha.\;\beta:\U_{\ell'}\quad\Delta\vdash e_2:\alpha:\U_{\ell_1}
\end{matrix}}{\Delta\vdash e_1\;e_2:\beta[e_2/x]:\U_{\ell_2}}\qquad
\frac{\Delta,x:\alpha:\U_{\ell_1}\vdash e:\beta:\U_{\ell_2}}{\Delta\vdash\lambda x:\alpha.\;e:\forall x:\alpha.\;\beta:\U_{\imax(\ell_1,\ell_2)}}$$
$$\frac{\Delta,x:\alpha:\U_{\ell_1}\vdash \beta:\U_{\ell_2}\type}{\Delta\vdash\forall x:\alpha.\;\beta:\U_{\imax(\ell_1,\ell_2)}\type}$$
$$\frac{\Delta\vdash e:\alpha:\U_\ell\quad \Delta\vdash \beta:\U_{\ell'}\type\quad \Delta\vdash\alpha\equiv\beta:\U_\ell\quad\ell\equiv\ell'}{\Delta\vdash e:\beta:\U_{\ell'}}$$
In these rules $\Delta\vdash \alpha:\U_\ell\type$ is defined to be $\Delta\vdash \alpha:\U_\ell:\U_{S\ell}$. We also introduce a definitional equality judgment with universe contexts, with a built-in typing of both sides as well:

$$\boxed{\Delta\vdash e\equiv e':\alpha}$$
$$\frac{\Delta\vdash e:\alpha:\U_\ell}{\Delta\vdash e\equiv e:\alpha}\qquad
\frac{\Delta\vdash e\equiv e':\alpha}{\Delta\vdash e'\equiv e:\alpha}\qquad
\frac{\Delta\vdash e_1\equiv e_2:\alpha\quad\Delta\vdash e_2\equiv e_3:\alpha}{\Delta\vdash e_1\equiv e_3:\alpha}$$
$$\frac{\ell\equiv\ell'}{\vdash \U_\ell\equiv\U_{\ell'}:\U_{S\ell}}\qquad
\frac{\Delta\vdash e_1\equiv e'_1:\forall x:\alpha.\;\beta\quad \Delta\vdash e_2\equiv e'_2:\alpha}{\Delta\vdash e_1\;e_2\equiv e'_1\;e'_2:\beta[e_2/x]}$$
$$\frac{\Delta\vdash\alpha\equiv \alpha':\U_{\ell_1}\quad \Delta,x:\alpha:\U_{\ell_1}\vdash e\equiv e':\beta}{\Delta\vdash\lambda x:\alpha.\;e\equiv \lambda x:\alpha'.\;e':\forall x:\alpha.\;\beta}$$
$$\frac{\Delta\vdash \alpha\equiv \alpha':\U_{\ell_1}\quad \Delta,x:\alpha:\U_{\ell_1}\vdash \beta\equiv \beta':\U_{\ell_2}}{\Delta\vdash \forall x:\alpha.\;\beta\equiv \forall x:\alpha'.\;\beta':\U_{\imax(\ell_1,\ell_2)}}$$
$$\frac{\Delta,x:\alpha:\U_\ell\vdash e:\beta:\U_{\ell'}\quad\Delta\vdash e':\alpha:\U_\ell}{\Delta\vdash (\lambda x:\alpha.\;e)\;e'\equiv e[e'/x]:\beta[e'/x]}\qquad
\frac{\Delta\vdash e:\forall y:\alpha.\;\beta:\U_\ell}{\Delta\vdash \lambda x:\alpha.\;e\;x\equiv e:\forall y:\alpha.\;\beta}$$
$$\frac{\Delta\vdash h:p:\P\quad \Delta\vdash h':p:\P}{\Delta \vdash h\equiv h':p}\qquad
\frac{\Delta\vdash e\equiv e':\alpha\quad \Delta\vdash\alpha\equiv\beta:\U_\ell}{\Delta\vdash e\equiv e':\beta}$$

Like $\Gamma\vdash e:\alpha$, we can define a simple context checking judgment:
$$\boxed{\vdash\Delta\ok}\qquad
\frac{}{\vdash\cdot\ok}\qquad
\frac{\Delta\vdash\alpha:\U_\ell\type}{\vdash\Delta,x:\alpha:\U_\ell\ok}$$

Let the notation $[\Delta]$ denotes the context obtained from $\Delta$ by omitting the universes:
$$[\cdot]=\cdot\qquad
[\Delta,x:\alpha:\U_\ell]=[\Delta],x:\alpha$$
And let $\Delta\equiv\Delta'$ mean that the contexts are equivalent except for changing types and levels to definitionally equal ones:
$$\boxed{\Delta\equiv\Delta'}\qquad\frac{}{\cdot\equiv\cdot}\qquad
\frac{\Delta\equiv\Delta'\quad\ell\equiv\ell'\quad\Delta\vdash\alpha\equiv\beta:\U_\ell}{\Delta,x:\alpha:\U_\ell\equiv\Delta',x:\beta:\U_{\ell'}}$$

\begin{lemma}[Substitution and regularity for sort inference]
\begin{enumerate}
\item If $\Delta\vdash e:\alpha:\U_\ell$, then $\vdash \Delta\ok$.
\item If $\Delta\vdash \alpha:\U_\ell\type$, then $\vdash \Delta\ok$.
\item If $\Delta,x:\alpha:\U_\ell\vdash e_1:\beta:\U_{\ell'}$ and $\Delta\vdash e_2:\alpha:\U_\ell$,\\
then $\Delta\vdash e_1[e_2/x]:\beta[e_2/x]:\U_{\ell'}$.
\item If $\Delta,x:\alpha:\U_\ell\vdash e_1\equiv e_1':\beta$ and $\Delta\vdash e_2:\alpha:\U_\ell$,\\
then $\Delta\vdash e_1[e_2/x]\equiv e_1'[e_2/x]:\beta[e_2/x]$.
\item If $\Delta,x:\alpha:\U_\ell\vdash e_1:\beta:\U_\ell$ and $\Delta\vdash e_2\equiv e_2':\alpha$,\\
then $\Delta\vdash e_1[e_2/x]\equiv e_1[e_2'/x]:\beta[e_2/x]$.
\item If $\Delta,x:\alpha:\U_\ell\vdash \beta:\U_{\ell'}\type$ and $\Delta\vdash e_2:\alpha:\U_\ell$, then $\Delta\vdash \beta[e_2/x]:\U_{\ell'}\type$.
\item If $\Delta\vdash e:\alpha:\U_\ell$, then $\Delta\vdash \alpha:\U_\ell\type$.
\item If $\Delta\equiv\Delta'$ then $\vdash\Delta\ok$.
\item $\Delta\equiv\Delta'$ is an equivalence relation.
\item If $\Delta\equiv\Delta'$ and $\Delta\vdash e:\alpha:\U_\ell$ then $\Delta'\vdash e:\alpha:\U_\ell$.
\item If $\Delta\equiv\Delta'$ and $\Delta\vdash e\equiv e':\alpha$ then $\Delta'\vdash e\equiv e':\alpha$.
\item If $\Delta\vdash e:\alpha:\U_\ell$, then $[\Delta]\vdash e:\alpha$.
\item If $\Delta\vdash e:\alpha:\U_\ell$, then $[\Delta]\vdash \alpha:\U_\ell$.
\item If $\Delta\vdash \alpha:\U_\ell\type$, then $[\Delta]\vdash \alpha:\U_\ell$.
\item If $\vdash\Delta\ok$, then $\vdash[\Delta]\ok$.
\item If $\Delta\equiv\Delta'$ then $[\Delta]\equiv[\Delta']$.
\end{enumerate}
\end{lemma}
\begin{proof}\ \\[-7mm]
\begin{itemize}
\item Part (1) an easy induction. Part (2) is a special case of (1).
\item (3) is by induction on the first premise, using the substitution property and (3) in the conversion rule.
\item (8) is a special case of (7).
\item (9) is an easy induction, using (8) in the application case.
\item (10) is by cases on $\Delta$.
\item UNFINISHED
\end{itemize}
\end{proof}

\begin{lemma}[Properties of the context truncation]
\begin{enumerate}
\item If $\Delta\vdash e:\alpha:\U_\ell$, then $[\Delta]\vdash e:\alpha$.
\item If $\Delta\vdash e:\alpha:\U_\ell$, then $[\Delta]\vdash \alpha:\U_\ell$.
\item If $\Delta\vdash \alpha:\U_\ell\type$, then $[\Delta]\vdash \alpha:\U_\ell$.
\item If $\vdash\Delta\ok$, then $\vdash[\Delta]\ok$.
\item If $\Delta\equiv\Delta'$ then $[\Delta]\equiv[\Delta']$.
\end{enumerate}
\end{lemma}
\begin{proof}\ \\[-7mm]
\begin{itemize}
\item (1) and (2) are proven together by induction. The substitution property is used in the application case.
\item (3) is a special case of (1), (4) is by cases using (3).
\end{itemize}
\end{proof}


\begin{lemma}[Sort inference]
\begin{enumerate}
\item If $\Gamma\vdash e:\alpha$, then $\Delta\vdash e:\alpha:\U_\ell$ for some $\Delta,\ell$ such that $[\Delta]=\Gamma$.
\item If $\Gamma\vdash \alpha:\U_\ell$, then $\Delta\vdash \alpha:\U_\ell\type$ for some $\Delta$ such that $[\Delta]=\Gamma$.
\item If $\Gamma\vdash e\equiv e'$, then $\Delta\vdash e\equiv e':\alpha$ for some $\Delta,\alpha$ such that $[\Delta]=\Gamma$.
\end{enumerate}
\end{lemma}
\begin{proof}
All parts are proven by mutual induction.
\begin{itemize}
\item UNFINISHED
\end{itemize}
\end{proof}
The key to why this judgment is relevant is the observation that the conversion rule cannot do anything interesting at the universe level. We want to use this to prove that $\Gamma\vdash\U_\ell\equiv\U_{\ell'}$ implies $\ell\equiv\ell'$.

UNFINISHED

\subsection{Unique typing}
There are a large number of ``natural'' properties about the typing and definitional equality judgments we will want to be true in order to reason that certain judgments are not derivable for ``obvious'' reasons, for example it is not possible to prove $\vdash\P:\P$ (which is a necessary condition for soundness).

\begin{theorem}[Unique typing]\label{unique}
\begin{enumerate}
\item If $\Gamma\vdash e:\alpha$ and $\Gamma\vdash e:\beta$, then $\Gamma\vdash\alpha\equiv\beta$.
\item If $\Gamma\vdash x\equiv y$, then $x=y$.
\item If $\Gamma\vdash \U_\ell\equiv\U_{\ell'}$, then $\ell\equiv\ell'$.
\item If $\Gamma\vdash \forall x:\alpha.\;e\equiv\forall x:\alpha'.\;e'$, then $\Gamma\vdash \alpha\equiv\alpha'$ and $\Gamma,x:\alpha\vdash e\equiv e'$.
\item If $\Gamma\vdash c_{\ell}\equiv c_{\ell'}$ (where $c$ is an axiomatic constant like $\mathsf{choice}$ or an inductive type or constructor, not a definition), then $\ell\equiv\ell'$.
\item If $\Gamma\vdash \lambda x:\alpha.\;e\equiv\lambda x:\alpha'.\;e'$, then $\Gamma\vdash \alpha\equiv\alpha'$ and $\Gamma,x:\alpha\vdash e\equiv e'$.
\end{enumerate}
\end{theorem}

Unfortunately, we cannot yet prove this theorem. The critical step is the Church-Rosser theorem, which we will develop in the next section. However, we can set up the induction, which is necessary now since the Church-Rosser theorem will require that this theorem is true, and we will be caught in a circularity unless we are careful about the claims.

We will prove this theorem by induction on the number of alternations between the judgments $\Gamma\vdash e:\alpha$ and $\Gamma\vdash\alpha\equiv\beta$ (which are mutually recursive). Define $\Gamma\vdash_n e:\alpha$ and $\Gamma\vdash_m\alpha\equiv\beta$ by induction on $n\in\N$ as follows:
\begin{itemize}
\item $\Gamma\vdash_0\alpha\equiv\beta$ iff $\alpha=\beta$.
\item $\Gamma\vdash_{n+1}\alpha\equiv\beta$ iff there is a proof of $\Gamma\vdash\alpha\equiv\beta$ using  only $\Gamma\vdash_n e:\alpha$ typing judgments.
\item Assuming $\Gamma\vdash_m\alpha\equiv\beta$ is defined for $m\le n$, $\Gamma\vdash_n e:\alpha$ means that there is a proof of $\Gamma\vdash e:\alpha$ in which all appeals to the conversion rule use $\Gamma\vdash_m\alpha\equiv\beta$ for $m\le n$.
\end{itemize}
So if $\Gamma\vdash_0 e:\alpha$, then there is a proof that does not use the conversion rule at all; if $\Gamma\vdash_1\alpha\equiv\beta$ then there is a proof whose typing judgments do not use the conversion rule; if $\Gamma\vdash_1 e:\alpha$ then there is a proof using only the $1$-provable conversion rule; and so on. We will prove theorem \ref{unique} by induction on this $n$.

\begin{lemma}[$n$-provability basics]
\begin{enumerate}
\item If $m\le n$ then $\Gamma\vdash_m e:\alpha$ implies $\Gamma\vdash_n e:\alpha$.
\item If $m\le n$ then $\Gamma\vdash_m\alpha\equiv\beta$ implies $\Gamma\vdash_n \alpha\equiv\beta$.
\item If $\Gamma\vdash e:\alpha$ then $\Gamma\vdash_n e:\alpha$ for some $n\in\N$.
\item If $\Gamma\vdash\alpha\equiv\beta$ then $\Gamma\vdash_n \alpha\equiv\beta$ for some $n\in\N$.
\end{enumerate}
\end{lemma}
\begin{proof}
(1) is immediate from the definition, (2) follows from (1). (3,4) are proven by a mutual induction on the typing judgment.
\end{proof}
\begin{definition}
Say that $\vdash_n$ has \emph{unique typing} if the following properties hold:
\begin{enumerate}
\item If $\Gamma\vdash_n e:\alpha$ and $\Gamma\vdash_n e:\beta$, then $\Gamma\vdash_n\alpha\equiv\beta$.
\item If $\Gamma\vdash_n x\equiv y$, then $x=y$.
\item If $\Gamma\vdash_n \U_\ell\equiv\U_{\ell'}$, then $\ell\equiv\ell'$.
\item If $\Gamma\vdash_n \forall x:\alpha.\;e\equiv\forall x:\alpha'.\;e'$, then $\Gamma\vdash_n \alpha\equiv\alpha'$ and $\Gamma,x:\alpha\vdash_n e\equiv e'$.
\item If $\Gamma\vdash_n c_{\ell}\equiv c_{\ell'}$ (where $c$ is an axiomatic constant like $\mathsf{choice}$ or an inductive type or constructor, not a definition), then $\ell\equiv\ell'$.
\item If $\Gamma\vdash_n \lambda x:\alpha.\;e\equiv\lambda x:\alpha'.\;e'$, then $\Gamma\vdash_n \alpha\equiv\alpha'$ and $\Gamma,x:\alpha\vdash_n e\equiv e'$.
\end{enumerate}
\end{definition}
Clearly, it suffices to prove that $\vdash_n$ has unique typing for every $n$ to establish theorem \ref{unique}. We can show the base case:
\begin{lemma}
$\vdash_0$ has unique typing.
\end{lemma}
\begin{proof}
By the weakening lemma, we can use instead the judgment $\vdash'_0$ which has no weakening rule and no conversion rule, and then for each $e$ there is exactly one applicable rule. Thus we can prove by induction on $\Gamma\vdash'_0 e:\alpha$ and inversion on $\Gamma\vdash'_0 e:\beta$ that $\alpha=\beta$.
\end{proof}

\subsection{The Church-Rosser theorem}
\emph{Note: In this section, we will omit the indices from the provability relation, but we will focus on characterizing the $\equiv$ relation at a particular level. So read $\Gamma\vdash \alpha\equiv\beta$ as $\Gamma\vdash_{n+1} \alpha\equiv\beta$, and $\Gamma\vdash e:\alpha$ as $\Gamma\vdash_n e:\alpha$. Also (and importantly) we will assume that $\vdash_n$ has unique typing, which will prevent the appearance of certain pathologies.}

The standard formulation of the Church-Rosser theorem, when applied to the $\rightsquigarrow$ reduction relation, is not true; under reasonable definitions of reduction, Lean will not have unique normal forms, because of proof irrelevance. (We already saw how this plays out in section \ref{undecidable}). All other substantive reduction rules act on terms the same way regardless of their types. To analyze this, we will split the definitional equality judgment into two parts: A $\beta\eta\delta\zeta\iota$-reduction relation (henceforth abbreviated $\kappa$ reduction), and a relation that does proof irrelevance. The idea is that $\kappa$ reduction satisfies a modified version of the Church-Rosser theorem, while proof irrelevance picks up the pieces, quantifying exactly how non-unique the normal form is.

The $\kappa$ reduction relation is defined with compatibility rules such as these for every syntax operator:
$$\boxed{\Gamma\vdash e\rightsquigarrow_\kappa e'}\qquad
\frac{\Gamma\vdash e_1 \rightsquigarrow_\kappa e_1'}{\Gamma\vdash e_1\;e_2\rightsquigarrow_\kappa e_1'\;e_2}\qquad
\frac{\Gamma\vdash e_2 \rightsquigarrow_\kappa e_2'}{\Gamma\vdash e_1\;e_2\rightsquigarrow_\kappa e_1\;e_2'}$$
$$\frac{\Gamma\vdash \alpha \rightsquigarrow_\kappa\alpha'}{\Gamma\vdash \lambda x:\alpha.\;e\rightsquigarrow_\kappa \lambda x:\alpha'.\;e}\qquad
\frac{\Gamma,x:\alpha\vdash e \rightsquigarrow_\kappa e'}{\Gamma\vdash \lambda x:\alpha.\;e\rightsquigarrow_\kappa \lambda x:\alpha.\;e'}\qquad\dots$$
The substantive rules are:
$$\frac{}{\Gamma\vdash (\lambda x:\alpha.\;e)\;e'\rightsquigarrow_\kappa e[e'/x]}\qquad
\frac{\begin{matrix}
\Gamma\vdash e:\forall x:\alpha.\;\beta\quad e\mbox{ is not a lambda}\\
\Gamma,x:\alpha\vdash h\equiv_p x\quad \Gamma,x:\alpha\vdash e\;h\rightsquigarrow_\kappa e'
\end{matrix}}{\Gamma\vdash e\rightsquigarrow_\kappa\lambda x:\alpha.\;e'}$$
$$\frac{\mathsf{def}\;c:\alpha:=e}{\Gamma\vdash c\rightsquigarrow_\kappa e}\qquad
\frac{}{\Gamma\vdash \elet{x:\alpha:=e'}{e}\rightsquigarrow_\kappa e[e'/x]}$$
$$\frac{P\mbox{ is inductive with constructor }c}{\Gamma\vdash \rec_P\;C\;e\;p\;(c\;b)\rightsquigarrow_\kappa e_c\;b\;v}\qquad
\frac{}{\Gamma\vdash \lift_R\;f\;h\;(\mk_R\;a)\rightsquigarrow_\kappa f\;a}$$

See section \ref{inductive} for the variable names and types used in the $\iota$ rule. There is no $\eta$ reduction rule here; instead we have a ``lambda-forming'' extensionality principle (we will call it the $\epsilon$ rule), which also mixes in a bit of proof irrelevance. To see what can go wrong without this, consider the following reductions, using $\rec_{a=}:\forall C.\;C\;a\to \forall b.\;a=b\to C\;b$:
\begin{align*}
&\lambda h:a=a.\;\rec_{a=}\;C\;e\;a\;h\rightsquigarrow_\eta\rec_{a=}\;C\;e\;a\\
&\lambda h:a=a.\;\rec_{a=}\;C\;e\;a\;h\rightsquigarrow_\iota\lambda h:a=a.\;e
\end{align*}

We want to say that $\rec_{a=}\;C\;e\;a$ still reduces to $\lambda h:a=a.\;e$ even though it is not fully applied. To do this, we apply extensionality:
$$\frac{\refl\;a\equiv_p h\quad\rec_{a=}\;C\;e\;a\;(\refl\;a)\rightsquigarrow_\kappa e}{\rec_{a=}\;C\;e\;a\rightsquigarrow_\kappa \lambda h:a=a.\;e}$$


The proof irrelevance relation deals with all the ways that normal forms can fail to be unique. Specifically, this relation is responsible for changing universe levels and changing proofs.
$$\boxed{\Gamma\vdash e\equiv_p e'}$$
$$\frac{\Gamma\vdash e:\alpha}{\Gamma\vdash e\equiv_p e}\qquad
\frac{\Gamma\vdash\alpha\equiv_p\alpha'\quad \Gamma,x:\alpha\vdash e\equiv_p e'}{\Gamma\vdash \forall x:\alpha.\;e\equiv_p \forall x:\alpha'.\;e'}\qquad
\frac{\Gamma\vdash e_1\equiv_p e_1'\quad \Gamma\vdash e_2\equiv_p e_2'}{\Gamma\vdash e_1\;e_2\equiv_p e_1'\;e_2'}\qquad\dots$$
$$\frac{\ell\equiv\ell'}{\vdash \U_\ell\equiv_p\U_{\ell'}}\qquad
\frac{\Gamma\vdash\alpha\equiv\alpha'\quad \Gamma,x:\alpha\vdash e\equiv_p e'}{\Gamma\vdash \lambda x:\alpha.\;e\equiv_p \lambda x:\alpha'.\;e'}\qquad
\frac{\Gamma\vdash p:\P\quad \Gamma\vdash h:p\quad \Gamma\vdash h':p}{\Gamma \vdash h\equiv_p h'}$$
Note also that this relation allows full definitional equality in the domain of a lambda, that is, it requires that $\Gamma\vdash \alpha\equiv\alpha'$ instead of $\Gamma\vdash \alpha\equiv_p\alpha'$ like the other rules. To give a hint on why this is relevant, consider the following reduction sequence:

\begin{align*}
&\lambda x:\alpha.\;(\lambda y:\beta.\;e)\;x\rightsquigarrow_\eta\lambda y:\beta.\;e\\
&\lambda x:\alpha.\;(\lambda y:\beta.\;e)\;x\rightsquigarrow_\beta\lambda x:\alpha.\;e[x/y]=\lambda y:\alpha.\;e
\end{align*}
(We view $\alpha$-equivalent terms as syntactically equal.) By reducing the expression in two ways, we obtain the ``same'' expression $\lambda y.\;e$, but the type annotation is different, and although we know from the application rule that $\alpha\equiv\beta$, this equivalence may involve the full strength of the type system to prove. These two terms are, however, $\equiv_p$-equivalent, and this is close enough to equal for us to prove the main theorems. (We don't actually use the $\eta$ reduction relation, but the $\epsilon$ reduction also demonstrates this: the type $\alpha$ on the right of the rule is only constrained by the typing of $e$, so we can use anything definitionally equal to $\alpha$ instead.)

\begin{lemma}[Regularity of reductions]
\begin{enumerate}
\item If $\Gamma\vdash e:\alpha$ and $\Gamma\vdash e\rightsquigarrow_\kappa e'$, then $\Gamma\vdash e\equiv e':\alpha$.
\item $\equiv_p$ is an equivalence relation.
\item If $\Gamma\vdash e\equiv_p e'$, then $\Gamma\vdash e\equiv e'$.
\item\label{p_subst} If $\Gamma,x:\alpha\vdash e_1\equiv_p e_1'$ and $\Gamma\vdash e_2\equiv_p e_2'$ then $\Gamma\vdash e_1[e_2/x]\equiv_p e_1'[e_2'/x]$.
\end{enumerate}
\end{lemma}
\begin{proof}
All parts are easy inductions.
\end{proof}
Note that the first part implies subject reduction for $\rightsquigarrow_\kappa$.

\begin{theorem}[Church-Rosser property]\label{church_rosser}
If $\Gamma\vdash e:\alpha$, and $e\rightsquigarrow_\kappa^* e_1$ and $e\rightsquigarrow_\kappa^* e_2$, then there exists $e_1'$ and $e_2'$ such that $\Gamma\vdash e_1'\equiv_p e_2'$, and $e_1\rightsquigarrow_\kappa^* e_1'$ and $e_2\rightsquigarrow_\kappa^* e_2'$.
\end{theorem}
The proof follows the Tait--Martin-L\"{o}f method, extended to all the $\kappa$ rules. Define the parallel reduction $\gg_\kappa$ by the following rules:
$$\frac{}{\Gamma\vdash x\gg_\kappa x}\qquad
\frac{\Gamma\vdash \alpha\gg_\kappa\alpha'\quad \Gamma\vdash e\gg_\kappa e'}{\Gamma\vdash \lambda x:\alpha.\;e\gg_\kappa \lambda x:\alpha'.\;e'}\qquad
\frac{\Gamma\vdash e_1\gg_\kappa e_1'\quad \Gamma\vdash e_2\gg_\kappa e_2'}{\Gamma\vdash e_1\;e_2\gg_\kappa e_1'\;e_2'}\qquad\dots$$
$$\frac{\Gamma\vdash e_1\gg_\kappa e_1'\quad \Gamma\vdash e_2\gg_\kappa e_2'}{\Gamma\vdash (\lambda x:\alpha.\;e_1)\;e_2\gg_\kappa e_1'[e_2'/x]}\qquad
\frac{x\notin FV(e)\quad \Gamma,x:\alpha\vdash h\equiv_p x\quad \Gamma\vdash e\gg_\kappa e'}{\Gamma\vdash \lambda x:\alpha.\;e\;h\gg_\kappa e'}$$
$$\frac{\Gamma\vdash e_1\gg_\kappa e_1'\quad e_2\gg_\kappa e_2'}{\Gamma\vdash \elet{x:\alpha:=e_1}{e_2}\gg_\kappa e_2'[e_1'/x]}\qquad
\frac{\mathsf{def}\;c:\alpha:=e\quad \Gamma\vdash e\gg_\kappa e'}{\Gamma\vdash c\gg_\kappa e'}$$
$$\frac{\begin{matrix}
q\equiv_p \mk_R\;a\quad \lift_R\;f\;h\;q=k\;\bar x\\
\Gamma\vdash f\gg_\kappa f'\quad \Gamma\vdash a\gg_\kappa a'
\end{matrix}}{\Gamma\vdash k\gg_\kappa f'\;a'}\qquad
\frac{\begin{matrix}
P\mbox{ is inductive with constructor }c\\
h\equiv_p c\;b\quad
\rec_P\;C\;e\;p\;h=k\;\bar x\\
C\gg_\kappa C'\quad e\gg_\kappa e'\quad b\gg_\kappa b'\quad p\gg_\kappa p'
\end{matrix}}{k\gg_\kappa\lambda\bar x.\; e_c'\;b'\;v'}$$
The ellipsis on the first line abbreviates compatibility rules for all the term constructors, recursing into all subterms like in the examples for lambda and application. All the substantive rules also follow a similar pattern: for each substantive rule in $\rightsquigarrow_\kappa$, there is a corresponding rule where after applying the $\rightsquigarrow_\kappa$ rule all variables on the RHS are $\gg_\kappa$ evaluated to the primed versions, and these are what end up in the target expression. (Note that in the $\iota$ rule, $v$ is a term that mentions $e$ and $p$; these are replaced by the primed versions in $v'$. Again, to avoid notational overhead, the typing contexts in the $\iota$ rule have been omitted.)

In addition, we define the following ``complete reduction'' $\Gamma\vdash e\ggg_\kappa e'$ by exactly the same rules as $\gg_\kappa$, except that the compatibility rules only apply if none of the substantive rules are applicable, and the sequence $\bar x$ in the $\iota$ rules is as short as possible (so $k$ is the largest subterm for which the $\iota$ rule still applies). This makes $\ggg_\kappa$ almost deterministic (producing a unique $e'$ given $e$), except that the $\equiv_p$ hypothesis in the $\iota$ rule allows some freedom of choice of the parameters $b$.

It is easy to prove the following properties by induction:
\begin{lemma}[Properties of $\gg_\kappa$]\label{gg_prop}
\begin{enumerate}
\item $e\gg_\kappa e$.
\item\label{red_gg} If $e\rightsquigarrow_\kappa e'$ then $e\gg_\kappa e'$.
\item\label{gg_red} If $e\gg_\kappa e'$ then $e\rightsquigarrow_\kappa^* e'$.
\item\label{gg_subst} If $e_1\gg_\kappa e_1'$ and $e_2\gg_\kappa e_2'$ then $e_1[e_2/x]\gg_\kappa e_1'[e_2'/x]$.
\item\label{ggg_ex} If $\Gamma\vdash e:\alpha$, then $\Gamma\vdash e\ggg_\kappa e'$ for some $e'$.
\end{enumerate}
\end{lemma}

\begin{lemma}[Compatibility of $\gg_\kappa$ with $\equiv_p$]\label{gg_compat}
If $\Gamma\vdash e_1\equiv_p e_3\gg_\kappa e_2$, then there exists $e_4$ such that $\Gamma\vdash e_1\gg_\kappa e_4\equiv_p e_2$.
\end{lemma}
\begin{proof}
By induction on $e_1\equiv_p e_3$ and inversion on $e_3\gg_\kappa e_2$.
\begin{itemize}
\item If $e_1\equiv_p e_3=e_1$ by the reflexivity rule, then $e_1\gg_\kappa e_2\equiv_p e_2$.
\item If $e_1\equiv_p e_3$ by the proof irrelevance rule, then $e_3:p:\P$, so $e_2:p:\P$ as well and hence $e_1\gg_\kappa e_1\equiv_p e_2$.
\item If $e_1\equiv_p e_3$ and $e_3\gg_\kappa e_2$ both use the same compatibility rule, then it is immediate from the induction hypothesis.
\item If $e_1:p:\P$ is a proof, then $e_1\gg_\kappa e_1\equiv_p e_2$. (We will thus assume that $e_1$ is not a proof in later cases.)
\item If $\lambda x:\alpha_1.\;e_1\equiv_p\lambda x:\alpha_3.\;e_3\gg_\kappa\lambda x:\alpha_2.\;e_2$ by the lambda compatibility rule, then $\alpha_1\equiv\alpha_3\gg_\kappa\alpha_2$ and $e_1\equiv_p e_3\gg_\kappa e_2$, and by the IH we have $e_1\gg_\kappa e_4\equiv_p e_2$, so $\lambda x:\alpha_1.\;e_1\gg_\kappa\lambda x:\alpha_3.\;e_4\equiv_p\lambda x:\alpha_2.\;e_2$.
\item If $(\lambda x:\alpha_1.\;e_1)\;e_1'\equiv_p(\lambda x:\alpha_3.\;e_3)\;e_3'\gg_\kappa e_2[e_2'/x]$ where $e_1\equiv_p e_2\gg_\kappa e_3$, $e_1'\equiv_p e_2'\gg_\kappa e_3'$ and $\alpha_1\equiv\alpha_3$, then $(\lambda x:\alpha_1.\;e_1)\;e_1'\gg_\kappa e_1[e_1'/x]\equiv_p e_2[e_2'/x]$. (Other cases are similar, when the $\equiv_p$ is proven by compatibility rules and the $\gg_\kappa$ is a substantive rule.)
\item If $\lambda x:\alpha_1.\;e_1\;h_1\equiv_p\lambda x:\alpha_3.\;e_3\;h_3\gg_\kappa e_2$ where $e_1\equiv_p e_3\gg_\kappa e_2$ and $h_1\equiv_p h_3\equiv_p x$, then by the IH $e_1\gg_\kappa e_4\equiv_p e_2$ so that $\lambda x:\alpha_1.\;e_1\;h_1\gg_\kappa e_4\equiv_p e_2$.
\item It is not possible to have $e_1\equiv_p\lift_R\;f\;h\;q$, $q\equiv_p \mk_R\;a$ by proof irrelevance at any intermediate stage, because then $\beta:\P$ (where $f:\alpha\to\beta$) so $e_1$ is a proof. (\textsc{Warning}: this step requires some justification, see section \ref{technical})
\item If $k_1\equiv_p k_3\gg_\kappa \lambda \bar x.\; (e_2)_c\;b_2\;v_2$ where $\rec_P\;C_3\;e_3\;p_3\;h_3=k_3\;\bar x$, then $\rec_P\;C_1\;e_1\;p_1\;h_1=k_1\;\bar x$ by repeated inversion with the application rule, so for each of the variables $q\in\{C,e,b,p\}$, $q_1\equiv_p q_3\gg_\kappa q_2$ implies $q_1\gg_\kappa q_4\equiv_p q_2$, so $k_1\gg_\kappa \lambda x.\; (e_4)_c\;b_4\;v_4\equiv_p\lambda x.\; (e_2)_c\;b_2\;v_2$.
\end{itemize}
\end{proof}
\begin{lemma}[Triangle lemma]\label{cr_lemma}
If $\Gamma\vdash e:\alpha$, $e\gg_\kappa e'$, and $e\ggg_\kappa e^\bullet$, then there exists $e^\circ$ such that $\Gamma\vdash e'\gg_\kappa e^\circ\equiv_p e^\bullet$.
\end{lemma}
\begin{proof}
By induction on $e\ggg_\kappa e^\bullet$ and inversion on $e\gg_\kappa e'$.
\begin{itemize}
\item If $x\lll_\kappa x\gg_\kappa x$, then $x\gg_\kappa x\equiv_p x$.
\item If $e\ggg_\kappa e^\bullet$ by the beta rule:
\begin{itemize}
\item If $e_1^\bullet[e_2^\bullet/x]\lll_\kappa(\lambda x:\alpha.\;e_1)\;e_2\gg_\kappa e_1'[e_2'/x]$ by the beta rule, then $e_1'\gg_\kappa e_1^\circ$ and $e_2'\gg_\kappa e_2^\circ$ by the inductive hypothesis, so $e_1'[e_2'/x]\gg_\kappa e_1^\circ[e_2^\circ/x]\equiv_p e_1^\bullet[e_2^\bullet/x]$ by the substitution property.
\item If $e_1^\bullet[e_2^\bullet/x]\lll_\kappa(\lambda x:\alpha.\;e_1)\;e_2\gg_\kappa (\lambda x:\alpha.\;e_1')\;e_2'$ by the application rule and lambda rule, then $(\lambda x:\alpha.\;e_1')\;e_2'\gg_\kappa e_1^\circ[e_2^\circ/x]\equiv_p e_1^\bullet[e_2^\bullet/x]$ by the beta rule for $\gg_\kappa$ and the IH.
\item If $e_1^\bullet\;h[e_2^\bullet/x]\lll_\kappa(\lambda x:\alpha.\;e_1\;h)\;e_2\gg_\kappa e_1'\;e_2'$ by the application rule and eta rule, then $e_1'\;e_2'\gg_\kappa e_1^\circ\;e_2^\circ\equiv_p e_1^\bullet\;h[e_2^\bullet/x]$ because $e_2^\circ\equiv_p e_2^\bullet=x[e_2^\bullet/x]\equiv_p h[e_2^\bullet/x]$.
\end{itemize}
\item If $e\ggg_\kappa e^\bullet$ by the eta rule:
\begin{itemize}
\item If $e^\bullet\lll_\kappa\lambda x:\alpha.\;e\;h\gg_\kappa e'$ by the eta rule, then $e'\gg_\kappa e^\circ\equiv_p e^\bullet$.
\item If $e^\bullet\lll_\kappa\lambda x:\alpha.\;e\;h\gg_\kappa \lambda x:\alpha'.\;e'\;h'$ by the lambda rule and application rule, then $\lambda x:\alpha'.\;e'\;h'\gg_\kappa e^\circ\equiv_p e^\bullet$.
\item If $\lambda y:\beta^\bullet.\;e^\bullet\lll_\kappa\lambda x:\alpha.\;(\lambda y:\beta.\;e)\;h\gg_\kappa \lambda x:\alpha'.\;e'[h/y]$ by the lambda rule and beta rule, then $\lambda x:\alpha'.\;e'[h/y]\gg_\kappa \lambda x:\alpha^\circ.\;e^\circ[h/y]\equiv_p$\\$\lambda x:\beta^\bullet.\;e^\bullet[x/y]=\lambda y:\beta^\bullet.\;e^\bullet$. (Here we have used the $\lambda$ rule for $\equiv_p$ to equate $\alpha^\circ\equiv\beta^\bullet$, which are def.eq. because $\alpha$ and $\beta$ are.)
\item If $\lift_R\;f^\bullet\;h^\bullet\lll_\kappa\lambda x:\alpha.\;\lift_R\;f\;h\;(\mk_R\;a)\gg_\kappa \lambda x:\alpha'.\;f'\;a'$ by the lambda rule and quotient $\iota$ rule, then $\mk_R\;a\equiv_p x$ and hence this is a quotient in $\P$; in this case $\lift_R\;f\;h$ is a proof, so $\lambda x:\alpha'.\;f'\;a'\gg_\kappa\lambda x:\alpha'.\;f'\;a'\equiv_p\lift_R\;f^\bullet\;h^\bullet$.
\end{itemize}
\item If $e^\bullet\lll_\kappa c\gg_\kappa e'$ by the delta rule, then $e'\gg_\kappa e^\circ\equiv_p e^\bullet$.
\item If $e\ggg_\kappa e^\bullet$ by the zeta rule:
\begin{itemize}
\item If $e_1^\bullet[e_2^\bullet/x]\lll_\kappa\elet{x:\alpha:=e_1}{e_2}\gg_\kappa e_2'[e_1'/x]$ by the zeta rule, then $e_1'[e_2'/x]\gg_\kappa e_1^\circ[e_2^\circ/x]\equiv_p e_1^\bullet[e_2^\bullet/x]$.
\item If $e_1^\bullet[e_2^\bullet/x]\lll_\kappa\elet{x:\alpha:=e_1}{e_2}\gg_\kappa \elet{x:\alpha':=e_1'}{e_2'}$ by the let compatibility rule, then $\elet{x:\alpha':=e_1'}{e_2'}\gg_\kappa e_1^\circ[e_2^\circ/x]\equiv_p e_1^\bullet[e_2^\bullet/x]$ by the zeta rule.
\end{itemize}
\item If $e\ggg_\kappa e^\bullet$ by the inductive iota rule:
\begin{itemize}
\item If $\lambda\bar x.\;e_c^\bullet\;b^\bullet\;v^\bullet\lll_\kappa k\gg_\kappa \lambda\bar x.\;e_c'\;b'\;v'$ by the iota rule, where $\rec_P\;C\;e\;p\;h=k\;\bar x$ and  then $\lambda\bar x.\;e_c'\;b'\;v'\gg_\kappa \lambda\bar x.\;e_c^\circ\;b^\circ\;v^\circ\equiv_p \lambda\bar x.\;e_c^\bullet\;b^\bullet\;v^\bullet$.
\item If $\rec_P\;C\;e\;p\;h=k\;\bar x\;\bar y$, and the $\ggg_\kappa$ reduction uses $k:=k\;\bar x$ and the $\gg_\kappa$ reduction reduces $k\;\bar x$ by using the application rule on $\bar x$ (which must result in $\bar x$ since variables cannot reduce) and then the iota rule on $k$, we have
$\lambda\bar y.\;e_c^\bullet\;b^\bullet\;v^\bullet\lll_\kappa k\;\bar x \gg_\kappa(\lambda\bar x\;\bar y.\;e_c'\;b'\;v')\;\bar x$, where $\rec_P\;C\;e\;p\;h=k\;\bar x\;\bar y$ and  then $\lambda\bar x.\;e_c'\;b'\;v'\gg_\kappa \lambda\bar x.\;e_c^\circ\;b^\circ\;v^\circ\equiv_p \lambda\bar x.\;e_c^\bullet\;b^\bullet\;v^\bullet$.
\item If $e^\bullet\lll_\kappa\lambda x:\alpha.\;e\;h\gg_\kappa \lambda x:\alpha'.\;e'\;h'$ by the lambda rule and application rule, then $\lambda x:\alpha'.\;e'\;h'\gg_\kappa e^\circ\equiv_p e^\bullet$.
\item If $\lambda y:\beta^\bullet.\;e^\bullet\lll_\kappa\lambda x:\alpha.\;(\lambda y:\beta.\;e)\;h\gg_\kappa \lambda x:\alpha'.\;e'[h/y]$ by the lambda rule and beta rule, then $\lambda x:\alpha'.\;e'[h/y]\gg_\kappa \lambda x:\alpha^\circ.\;e^\circ[h/y]\equiv_p$\\$\lambda x:\beta^\bullet.\;e^\bullet[x/y]=\lambda y:\beta^\bullet.\;e^\bullet$. (Here we have used the $\lambda$ rule for $\equiv_p$ to equate $\alpha^\circ\equiv\beta^\bullet$, which are def.eq. because $\alpha$ and $\beta$ are.)
\end{itemize}
\item If $e\ggg_\kappa e^\bullet$ by the inductive $\iota$ rule:
\begin{itemize}
\item If $e_1^\bullet[e_2^\bullet/x]\lll_\kappa\elet{x:\alpha:=e_1}{e_2}\gg_\kappa e_2'[e_1'/x]$ by the zeta rule, then $e_1'[e_2'/x]\gg_\kappa e_1^\circ[e_2^\circ/x]\equiv_p e_1^\bullet[e_2^\bullet/x]$.
\item If $e_1^\bullet[e_2^\bullet/x]\lll_\kappa\elet{x:\alpha:=e_1}{e_2}\gg_\kappa \elet{x:\alpha':=e_1'}{e_2'}$ by the let compatibility rule, then $\elet{x:\alpha':=e_1'}{e_2'}\gg_\kappa e_1^\circ[e_2^\circ/x]\equiv_p e_1^\bullet[e_2^\bullet/x]$ by the zeta rule.
\end{itemize}
\item If $e\ggg_\kappa e^\bullet$ by a compatibility rule:
\begin{itemize}
\item If $e_1^\bullet\;e_2^\bullet\lll_\kappa e_1\;e_2\gg_\kappa e_1'\;e_2'$ by the application rule, then $e_1'\;e_2'\gg_\kappa e_1^\circ\;e_2^\circ\equiv_p e_1^\bullet\;e_2^\bullet$.
\item If $\forall x:\alpha^\bullet.\;e^\bullet\lll_\kappa \forall x:\alpha.\;e\gg_\kappa \forall x:\alpha'.\;e'$ by the forall rule, then $\forall x:\alpha'.\;e'\gg_\kappa \forall x:\alpha^\circ.\;e^\circ\equiv_p \forall x:\alpha^\bullet.\;e^\bullet$.
\item Other compatibility rules follow the same pattern.
\end{itemize}
\end{itemize}
\end{proof}
The main proof of Church-Rosser is a corollary of lemma \ref{cr_lemma}, and does not differ substantially from the usual proof putting diamonds together, because the additional complication of having $\equiv_p$ at the bottom of the diamond commutes with all the other reductions.
\begin{proof}[of theorem \ref{church_rosser}]
We prove in succession the following theorems:
\begin{enumerate}
\item If $\Gamma\vdash e:\alpha$, and $e_1\ll_\kappa e\gg_\kappa e_2$, then $\exists e_1'\;e_2'.\; e_1\gg_\kappa e_1'\equiv_p e_2'\ll_\kappa e_2$.
\item If $\Gamma\vdash e:\alpha$, and $e_1\ll_\kappa^* e\gg_\kappa e_2$, then $\exists e_1'\;e_2'.\; e_1\gg_\kappa e_1'\equiv_p e_2'\ll_\kappa^* e_2$.
\item If $\Gamma\vdash e:\alpha$, and $e_1\ll_\kappa^* e\gg_\kappa^* e_2$, then $\exists e_1'\;e_2'.\; e_1\gg_\kappa^* e_1'\equiv_p e_2'\ll_\kappa^* e_2$.
\item If $\Gamma\vdash e:\alpha$, and $e_1\leftsquigarrow_\kappa^*e\rightsquigarrow_\kappa^* e_2$, then $\exists e_1'\;e_2'.\; e_1\rightsquigarrow_\kappa^* e_1'\equiv_p e_2'\leftsquigarrow_\kappa^* e_2$.
\end{enumerate}
(4) is the theorem we want.
\begin{enumerate}
\item If $e\gg_\kappa e_1,e_2$, then by lemma \ref{cr_lemma} there exists $e_1',e_2'$ such that $e_i\gg_\kappa e_i'$ and $e_1'\equiv_p e^\bullet\equiv_p e_2'$. But then $e_1'\equiv_p e_2'$ are as desired. 
\item By induction on $e\gg_\kappa^* e_1$. If $e\gg_\kappa^* e_1\gg_\kappa e_3$ and we have inductively that $e_1\gg_\kappa e_1'\equiv_p e_2'\ll_\kappa^* e_2$, then by applying (1) to $e_3\ll_\kappa e_1\gg_\kappa e_1'$ we obtain $e_3\gg_\kappa e_3'\equiv_p e_1''\ll_\kappa e_1'$, and by lemma \ref{gg_compat} applied to $e_1''\ll_\kappa e_1'\equiv_p e_2'$ we obtain $e_1''\equiv_p e_2''\ll_\kappa e_2'$ so that  $e_3\gg_\kappa e_3'\equiv_p e_1''\equiv_p e_2''\ll_\kappa e_2'\ll_\kappa^* e_2$.
\item By induction on $e\gg_\kappa^* e_2$. The proof is the same as (2), replacing lemma \ref{gg_compat} for the analogous statement for $\gg_\kappa^*$, i.e. if $e_1\gg_\kappa^*e_2$ and $e_1\equiv_p e_1'$ then there exists $e_2'$ such that $e_1'\gg_\kappa^*e_2'\equiv_p e_1'$. This follows by induction on lemma \ref{gg_compat}.
\item  The equivalence of (3) and (4) comes from properties \ref{red_gg} and \ref{gg_red} of lemma \ref{gg_prop}.
\end{enumerate}
\end{proof}

Now say that $\Gamma\vdash e_1\equiv_\kappa e_2$ if there exists $e_1',e_2'$ such that $e_1\rightsquigarrow_\kappa^* e_1'\equiv_p e_2'\leftsquigarrow_\kappa^* e_2$. This relation is obviously reflexive and symmetric and implies $\Gamma\vdash e_1\equiv e_2$, and the Church-Rosser property implies it is also transitive. Now we can completely characterize $\equiv$ as interlacing $\equiv_p$ and $\equiv_\kappa$ equivalence steps:

\begin{theorem}[Completeness of the $\kappa$ reduction]\label{ckappa}
If $\Gamma \vdash e:\alpha$, $\Gamma \vdash e':\alpha$, then
$\Gamma\vdash e\equiv e'$ if and only if $\Gamma\vdash e\;(\equiv_\kappa\cup\equiv_p)^*\;e'$, where $(\equiv_\kappa\cup\equiv_p)^*$ is the reflexive transitive closure of the union of relations $\Gamma\vdash e\equiv_\kappa e'$ and $\Gamma\vdash e \equiv_p e'$.
\end{theorem}
\begin{proof}
The reverse direction follows immediately from the fact that $\equiv$ is an equivalence relation. The forward direction is by induction on $\equiv$.
\begin{itemize}
\item The reflexivity and transitivity rules are immediate.
\item Since $\equiv_\kappa$ and $\equiv_p$ are equivalence relations, they are symmetric, so $(\equiv_\kappa\cup\equiv_p)^*$ is also symmetric and hence the symmetry rule holds.
\item For the compatibility rules, since both $\equiv_p$ and $\rightsquigarrow_\kappa$ have compatibility rules, this property passes to $\equiv_\kappa$ and $\equiv_\kappa\cup\equiv_p$ and $(\equiv_\kappa\cup\equiv_p)^*$. Thus, for example in the lambda case, we have $\Gamma\vdash\lambda x:\alpha.\;e\;(\equiv_\kappa\cup\equiv_p)^*\;\lambda x:\alpha.\;e'$ since $\Gamma,x:\alpha\vdash e\;(\equiv_\kappa\cup\equiv_p)^*\;\alpha'$ from the IH, and similarly $\Gamma\vdash\lambda x:\alpha.\;e\;(\equiv_\kappa\cup\equiv_p)^*\;\lambda x:\alpha.\;e'$, so by transitivity $\Gamma\vdash\lambda x:\alpha.\;e\;(\equiv_\kappa\cup\equiv_p)^*\;\lambda x:\alpha'.\;e'$.
\item The universe changing rules (for constants and $\U_\ell$) are in $\equiv_p$.
\item The $\beta$ and $\eta$ rules are in $\rightsquigarrow_\kappa$, and the proof irrelevance rule is in $\equiv_p$. All the other equivalence rules are also introduced in $\rightsquigarrow_\kappa$.
\end{itemize}
\end{proof}

This theorem lets us separate the dynamics of $\equiv$ into two parts, which are each well understood. This will give us a method to show when two expressions are \emph{not} equivalent.

\subsection{A technicality}\label{technical}
UNFINISHED
\subsection{Unique typing}

\begin{lemma}[Inversion of definitional equality]
\begin{enumerate}
\item If $\Gamma\vdash x\equiv y$, then $x=y$.
\item If $\Gamma\vdash \U_\ell\equiv\U_{\ell'}$, then $\ell\equiv\ell'$.
\item If $\Gamma\vdash \forall x:\alpha.\;e\equiv\forall x:\alpha'.\;e'$, then $\Gamma\vdash \alpha\equiv\alpha'$ and $\Gamma,x:\alpha\vdash e\equiv e'$.
\item If $\Gamma\vdash c_{\ell}\equiv c_{\ell'}$ (where $c$ is an axiomatic constant like $\mathsf{choice}$ or an inductive type or constructor, not a definition), then $\ell\equiv\ell'$.
\item If $\Gamma\vdash \lambda x:\alpha.\;e\equiv\lambda x:\alpha'.\;e'$, then $\Gamma\vdash \alpha\equiv\alpha'$ and $\Gamma,x:\alpha\vdash e\equiv e'$.
\end{enumerate}
\end{lemma}
\begin{proof}
For parts 1-4 the proof is the same, and we will illustrate it with part 3. Consider the class $\mathcal{C}$ of terms of the form $\forall x:\alpha'.\;e'$ where $\Gamma\vdash\alpha\equiv\alpha'$ and $\Gamma\vdash \alpha\equiv\alpha'$ and $\Gamma,x:\alpha\vdash e\equiv e'$. We know $\forall x:\alpha.\;e\in\mathcal{C}$ and wish to show $\forall x:\alpha'.\;e'\in\mathcal{C}$. By theorem \ref{ckappa}, it suffices to show that $\equiv_\kappa$ and $\equiv_p$ preserve membership in $\mathcal{C}$.
\begin{itemize}
\item The only $\rightsquigarrow_\kappa$ rule with a $\forall$ on the left is the compatibility rule, which also has a $\forall$ on the right and hence preserves elementhood in $\mathcal{C}$. That is, if $e\in\mathcal{C}$ and $e\in\mathcal{C}$ and UNFINISHED
\end{itemize}
\end{proof}

\begin{lemma}[Unique typing]
\begin{enumerate}
\item If $\Gamma\vdash e\equiv e'$ then there exists $\alpha$ such that $\Gamma\vdash e\equiv e':\alpha$, and for all $\alpha'$, if $\Gamma\vdash e:\alpha'$ or $\Gamma\vdash e':\alpha'$, then $\Gamma\vdash \alpha\equiv\alpha'$
\item If $\Gamma\vdash e:\alpha$ and $\Gamma\vdash e:\alpha'$, then $\Gamma\vdash \alpha\equiv\alpha'$.
\item If $\Gamma\vdash e:\forall x:\alpha.\;\beta$ and $\Gamma\vdash e:\forall x:\alpha.\;\beta'$, then $\Gamma,x:\alpha\vdash \beta\equiv\beta'$.
\item If $\Gamma\vdash e\equiv\U_\ell$ and $\Gamma\vdash e\equiv\U_{\ell'}$, then $\ell\equiv\ell'$.
\end{enumerate}
\end{lemma}
\begin{proof}
Note that the second part of (2) follows from the first: If $\Gamma\vdash e\equiv e'$, and $\Gamma\vdash e:\alpha$, then by regularity there exists $\alpha'$ such that $\Gamma\vdash e':\alpha'$, so $\Gamma\vdash\alpha\equiv\alpha'$ by (2), and hence $\Gamma\vdash e':\alpha$. The converse is similar.

All parts are proven by mutual induction on their first hypotheses. For part 1 of the theorem, we also invert the second hypothesis. If both come from the same typing rule:
\begin{itemize}
\item The weakening, variable, universe, forall, and lambda rules are all trivial.
\item In the application case we have $\Gamma\vdash e_1:\forall x:\alpha.\;\beta$, $\Gamma\vdash e_1:\forall x:\alpha'.\;\beta'$, $\Gamma\vdash e_2:\alpha$, $\Gamma\vdash e_2:\alpha'$, and from the inductive hypothesis we have $\Gamma\vdash\alpha\equiv\alpha'$ and $\Gamma\vdash\forall x:\alpha.\;\beta\equiv\forall x:\alpha'.\;\beta'$; we wish to show $\Gamma\vdash\beta[e_2/x]\equiv\beta'[e_2/x]$. In this case UNFINISHED
\end{itemize}
By induction on the respective judgments (all of the parts may be proven separately).
\end{proof}

This type system has unique typing in the sense that if $\Gamma\vdash e:\alpha$ and $\Gamma\vdash e:\beta$, then $\Gamma\vdash\alpha\equiv\beta$. But more useful than this is unique \emph{sorting}: If $\Gamma\vdash\alpha:\U_\ell$ and $\Gamma\vdash\alpha:\U_{\ell'}$, then $\ell\equiv\ell'$.

UNFINISHED

%\section{Reduction of inductive types to $\W$-types}

Given the complicated structure involved in simply stating the axioms of inductive types, one may wonder if there is an easier way. In fact there is; we can replace the whole structure of inductive types with a few simple inductive type constructors.

The most well known general form of our kind of inductive type is the $\W$-type, defined when $\Gamma\vdash A:\U_\ell$ and $\Gamma,x:A\vdash B:\U_\ell$:
$$\W x:A.\;B:=\mu w:\U_\ell.\;(\mathsf{sup}:\forall x:A.\;(B\to w)\to w)$$
This carries most of the ``power'' of inductive types, but we still need some glue to be able to reduce everything else to this. First, note that most of the telescopes $x::\alpha$ in an inductive type can be replaced by $\Sigma(x::\alpha)$, where $\Sigma ():=\bf 1$ and $\Sigma (x:\alpha,y::\beta):=\Sigma x:\alpha,\Sigma(y::\beta)$. This just packs up all the types in the telescope into one dependent tuple. Similarly, we want the types $\bf 0$ and $\alpha+\beta$ to pack up all the constructors into one.

To localize the universe management we will have a ``universe lift'' function $\ulift_u^v:\U_u\to\U_v$, defined when $u\le v$, as well as the $\nonempty$ operation (also known as the propositional truncation $\|\alpha\|$) to construct small eliminators. All the other type operators above will have the smallest possible universe level.

Finally, to handle inductive families and K-like eliminators, we will need the equality and $\acc$ types discussed previously. Here are the rules for these types:
\begin{align*}
e::=\dots&\mid\bot\mid\Sigma x:e.\;e\mid e+e\mid \ulift_{\ell}^{\ell}\;e\mid \|e\|\mid \mathsf{W} x:e.\;e\mid e=e\mid \acc_e\\
&\mid\rec_\bot\mid(e,e)\mid\pi_1\;e\mid\pi_2\;e\mid \inl e\mid \inr e\mid \rec_+\;e\;e\mid {\uparrow}e\mid {\downarrow}e\\
&\mid |e|\mid\rec_{||}\;e\mid\sup e\;e\mid\rec_\W\;e\mid \refl\;e\mid \rec_=\;e\;e\mid \intro_\acc\;e\;e\mid\rec_\acc\;e
\end{align*}
$$\frac{}{\vdash \bot:\P}\qquad
\frac{1\le\ell\quad \Gamma\vdash C:\U_\ell}{\Gamma\vdash \rec_\bot:\bot\to C}$$
$$\frac{\Gamma\vdash \alpha:\U_\ell\quad \Gamma,x:\alpha\vdash \beta:\U_{\ell'}}
{\Gamma\vdash \Sigma x:\alpha.\;\beta:\U_{\max(\ell,\ell',1)}}\qquad
\frac{\Gamma\vdash \alpha:\U_\ell\quad \Gamma\vdash \beta:\U_{\ell'}}
{\Gamma\vdash \alpha+\beta:\U_{\max(\ell,\ell',1)}}$$
$$\frac{\Gamma\vdash e_1:\alpha\quad \Gamma\vdash e_2:\beta\;e_1}
{\Gamma\vdash(e_1,e_2):\Sigma x:\alpha.\;\beta}\qquad
\frac{\Gamma\vdash p:\Sigma x:\alpha.\;\beta}{\Gamma\vdash \pi_1\;p:\alpha}\qquad
\frac{\Gamma\vdash p:\Sigma x:\alpha.\;\beta}
{\Gamma\vdash \pi_2\;p:\beta[\pi_1\;p/x]}$$
$$\frac{\Gamma\vdash \beta\type\quad \Gamma\vdash e:\alpha}{\Gamma\vdash \inl e:\alpha+\beta}\qquad
\frac{\Gamma\vdash \alpha\type\quad \Gamma\vdash e:\beta}{\Gamma\vdash \inr e:\alpha+\beta}$$
$$\frac{1\le\ell\quad \Gamma\vdash C:\alpha+\beta\to\U_\ell\quad \Gamma\vdash a:\forall x:\alpha.\;C\;(\inl x)\quad \Gamma\vdash b:\forall x:\beta.\;C\;(\inr x)}{\Gamma\vdash \rec_+\;a\;b:\forall p:\alpha+\beta.\;C\;p}$$
$$\frac{\Gamma\vdash \alpha:\U_\ell\quad \max(1,\ell)\le\ell'}
{\Gamma\vdash\ulift_{\ell}^{\ell'}\;\alpha:\U_{\ell'}}\qquad
\frac{\Gamma\vdash\ulift_{\ell}^{\ell'}\;\alpha:\U_{\ell'}\quad \Gamma\vdash e:\alpha}{\Gamma\vdash {\uparrow}e:\ulift_{\ell}^{\ell'}\;\alpha}\qquad
\frac{\Gamma\vdash e:\ulift_{\ell}^{\ell'}\;\alpha}{\Gamma\vdash {\downarrow}e:\alpha}$$
$$\frac{\Gamma\vdash \alpha\type}
{\Gamma\vdash\|\alpha\|:\P}\qquad
\frac{\Gamma\vdash e:\alpha}{\Gamma\vdash |e|:\|\alpha\|}\qquad
\frac{\Gamma\vdash C:\P\quad \Gamma\vdash f:\alpha\to C}{\Gamma\vdash \rec_{||}\;f:\|\alpha\|\to C}$$
$$\frac{\Gamma\vdash \alpha:\U_\ell\quad \Gamma,x:\alpha\vdash\beta:\U_{\ell'}}
{\Gamma\vdash\W x:\alpha.\;\beta:\U_{\max(\ell,\ell',1)}}\qquad
\frac{\Gamma\vdash a:\alpha\quad\Gamma\vdash f:\beta[a/x]\to \W x:\alpha.\;\beta}{\Gamma\vdash\sup a\;f:\W x:\alpha.\;\beta}$$
$$\frac{\begin{matrix}
1\le\ell\quad \Gamma\vdash C:(\W x:\alpha.\;\beta)\to\U_\ell\\
\Gamma\vdash e:\forall (a:\alpha)\;(f:\beta[a/x]\to\W x:\alpha.\;\beta).\;(\forall b:\beta[a/x].\;C\;(f\;b))\to C\;(\sup a\;f)
\end{matrix}}{\Gamma\vdash \rec_\W\;e:\forall w:(\W x:\alpha.\;\beta).\;C\;w}$$
$$\frac{\Gamma\vdash a:\alpha\quad\Gamma\vdash b:\alpha}
{\Gamma\vdash a=b:\P}\qquad
\frac{\Gamma\vdash a:\alpha}{\Gamma\vdash \refl\;a:a=a}$$
$$\frac{\Gamma\vdash a:\alpha\quad 1\le\ell\quad \Gamma\vdash C:\alpha\to\U_\ell\quad\Gamma\vdash e:C\;a}{\Gamma\vdash \rec_=\;e:\forall b:\alpha.\;a=b\to C\;b}$$
$$\frac{\Gamma\vdash r:\alpha\to\alpha\to\P}{\Gamma\vdash \acc_r:\alpha\to\P}\qquad
\frac{\Gamma\vdash x:\alpha\quad\Gamma\vdash f:\forall y:\alpha.\;r\;y\;x\to\acc_r\;y}{\Gamma\vdash \intro_\acc\;x\;f:\acc_r\;x}$$
$$\frac{\begin{matrix}
1\le\ell\quad \Gamma\vdash C:\alpha\to\U_\ell\\
\Gamma\vdash e:\forall x:\alpha.\; (\forall y:\alpha.\;r\;y\;x\to\acc_r\;y)\to (\forall y:\alpha.\;r\;y\;x\to C\;y)\to C\;x
\end{matrix}}{\Gamma\vdash \rec_\acc\;e:\forall x:\alpha.\;\acc_r\;x\to C\;x}$$
All of these could have been defined as inductive types in the sense of section \ref{inductive}:
\begin{align*}
\bot&:=\mu t:\P.\;0\\
\Sigma x:\alpha.\;\beta&:=\mu t:\U_{\max(\ell,\ell',1)}.\;(\mathsf{pair}:\forall x:\alpha.\;\beta\to t)\\
\alpha+\beta&:=\mu t:\U_{\max(\ell,\ell',1)}.\;(\mathsf{inl}:\alpha\to t)+(\mathsf{inr}:\beta\to t)\\
\ulift_{\ell}^{\ell'}\;\alpha&:=\mu t:\U_{\ell'}.\;(\mathsf{up}:\alpha\to t)\\
\|\alpha\|&:=\mu t:\P.\;(\intro:\alpha\to t)\\
\W x:\alpha.\;\beta&:=\mu t:\U_{\max(\ell,\ell',1)}.\;(\mathsf{sup}:\forall x:\alpha.\;(\beta\to t)\to t)\\
a=b&:=(\mu t:\alpha\to\P.\;(\refl:t\;a))\;b\\
\acc_r&:=\mu t:\alpha\to\P.\;(\intro:\forall x:\alpha.\;(\forall y:\alpha.\;r\;y\;x\to t\;y)\to t\;x)
\end{align*}
However, we are interested in taking them as primitive in this section and deriving general inductive types. All of the new operators have compatibility rules for $\equiv$ and $\Leftrightarrow$; we will not belabor this as they all look roughly the same: when all the parts are equivalent, so is the whole. For example:
$$\frac{\Gamma\vdash\alpha\equiv\alpha'\quad\Gamma,x:\alpha\vdash\beta\equiv\beta'}{\Gamma\vdash\Sigma x:\alpha.\;\beta\equiv\Sigma x:\alpha'.\;\beta'}$$

Since we will need to handle $\P$ specially in the proof of soundness, we have simplified all the large eliminating recursors to require $1\le\ell$. The general recursor can be constructed from this by using $C':=\lambda x:P.\;\ulift_{\ell}^{\max(1,\ell)}\;(C\;x)$ (for each such inductive type $P$).

In a few of the constructors, additional parameters are elided, such as $C$ in $\rec_\bot$; one should imagine that each constructor is sufficiently annotated to ensure unique typing. Following their interpretation as inductive types, they also come with the following $\iota$ rules:
\begin{align*}
\pi_1(a,b)&\equiv a\\
\pi_2(a,b)&\equiv b\\
\rec_+\;a\;b\;(\inl x)&\equiv a\;x\\
\rec_+\;a\;b\;(\inr x)&\equiv b\;x\\
{\downarrow\uparrow} x&\equiv x\\
\rec_\W\;e\;(\sup a\;f)&\equiv e\;a\;f\;(\lambda b:\beta[a/x].\;\rec_\W\;e\;(f\;b))\\
\rec_=\;e\;a\;h&\equiv e\\
\rec_\acc\;e\;x\;(\intro_\acc\;x\;f)&\equiv e\;x\;f\;(\lambda (y:\alpha)\;(h:r\;y\;x).\;\rec_\acc\;e\;y\;(f\;y\;h))
\end{align*}
which are valid in any context that typechecks everything on the LHS.

Here are a few additional type operators that can be defined from the ones given:
$${\bf 0}_\ell:=\ulift^\ell\;\bot\qquad\top:=\bot\to\bot\qquad{\bf 1}_\ell:=\ulift^\ell\;\top\qquad \alpha\times \beta:=\Sigma\_:\alpha.\;\beta$$
$$p\land q:=\|p\times q\|\qquad p\lor q:=\|p+q\|$$
$$\{x:\alpha\mid p\}:=\Sigma x:\alpha.\;p\qquad
\exists x:\alpha.\; p:=\|\{x:\alpha\mid p\}\|$$

The following additional ``$\eta$ rules'' are needed for the reduction, which are provable but not definitional equalities in Lean. Since we are going for soundness only, we will help ourselves to this modest strengthening of the system; moreover this is only for convenience -- without such $\eta$ rules we would only be able to go as far as indexed $\W$-types, which are more complex. (These rules are also required for this axiomatization since we've omitted the recursors in favor of projections for $\Sigma$ and $\ulift$.)
$${\uparrow\downarrow} x\equiv x\qquad (\pi_1\;x,\pi_2\;x)\equiv x$$
The results of section \ref{sec:unique} apply straightforwardly to this setting, with these two rules added as $\rightsquigarrow_\kappa$ reduction rules along with all the $\iota$ rules mentioned above.
%
\subsubsection{Translating type families}
Let us first suppose that the inductive family lives in a universe $1\le\ell$. In this case we don't have to worry about $\P$ and small elimination. The idea is to eliminate families by first erasing the indices to get a ``skeleton'' type $S$ that mixes all the different members of the family together, and then separately define a predicate $\mathsf{good}:S\to\forall x::\alpha.\P$ that carves out the members that actually belong to index $x$. The final result will be the type $\lambda x::\alpha.\;\{s:S\mid\mathsf{good}\;s\;x\}$. For example, the type
$$X=\mu t:\N\to\U_1.\;(\mathsf{one}:t\;1)+(\mathsf{double}:\forall n:\N.\;t\;n\to t\;(2n))$$
has the indices erased to get
$$S=\mu t:\U_1.\;(\mathsf{one}:t)+(\mathsf{double}:\forall n:\N.\;t\to t),$$
and then the predicate is defined by recursion on $S$:
\begin{align*}
\mathsf{good}\;\mathsf{one}\;m&:=m=1\\
\mathsf{good}\;(\mathsf{double}\;n\;x)\;m&:=m=2n\land \mathsf{good}\;x\;n
\end{align*}
Now $S$ will also be reduced to the $\W$-type:
$$S'=\W x:{\bf 1}+\N.\;\rec_+\;(\lambda\_.\;{\bf 0})\;(\lambda n.\;{\bf 1})$$
because there are two branches, one with no non-recursive arguments and one with a non-recursive argument of type $\N$ (hence ${\bf 1}+\N$), and first branch has no recursive arguments and the second has one.

So the general translation will take the form
$$P\;x\simeq\{s:\W p:A.\;B\;p\mid \rec_\W\;(\lambda (p:A)\; \_.\;G\;p)\;s\;x\},$$\vspace{-7mm}
\begin{align*}
\mbox{where}\qquad\Gamma&\vdash A:\U_\ell\\
\Gamma&\vdash B:A\to\U_\ell\\
\Gamma&\vdash G:\forall p:A.\;(B\;p\to\forall x::\alpha.\;\P)\to\forall x::\alpha.\;\P.
\end{align*}

We will construct these three terms recursively based on the derivation of the $\spec$ judgment.
$$\boxed{\Gamma;t:F\vdash K\spec\Rightarrow A;B;G}$$
$$\frac{1\le\ell\quad \Gamma\vdash x::\alpha}{\Gamma;t:\forall x::\alpha.\;\U_\ell\vdash 0\spec\Rightarrow {\bf 0};\rec_0;\rec_0}$$
$$\frac{\Gamma;t:F\vdash \beta\ctor\Rightarrow A_1;p.B_1;pgx.G_1\quad \Gamma;t:F\vdash K\spec\Rightarrow A;B;G}{\Gamma;t:F\vdash (c:\beta)+K\spec\Rightarrow A_1+A;\rec_+\;(\lambda p.B_1)\;B;\rec_+\;(\lambda pg\;(x::\alpha).\;G_1)\;G}$$
$$\boxed{\Gamma;t:F\vdash \beta\ctor\Rightarrow A;p.B;pgx.G}$$
$$\frac{\Gamma\vdash e::\alpha}{\Gamma;t:\forall x::\alpha.\;\U_\ell\vdash t\;e\ctor\Rightarrow {\bf 1}_\ell;\ p.\;{\bf 0}_\ell;\ pgx.\;x=e}$$
$$\frac{\Gamma\vdash \beta:\U_{\ell'}\quad \ell'\le\ell\quad \Gamma,y:\beta;t:\forall x::\alpha.\;\U_\ell\vdash\tau\ctor\Rightarrow A;p.B;pgx.G}{
\begin{array}{l}
\Gamma;t:\forall x::\alpha.\;\U_\ell\vdash\forall y:\beta.\;\tau\ctor\Rightarrow\Sigma y':\beta.\;A[y'/y];\\
\qquad p'.B[\pi_1\;p'/y][\pi_2\;p'/p];\ p'gx.G[\pi_1\;p'/y][\pi_2\;p'/p]
\end{array}}$$
$$\frac{\begin{matrix}
\Gamma\vdash \gamma::\U_{\ell'}\quad \Gamma,z::\gamma\vdash e::\alpha\quad \ell'_i\le\ell\\
\Gamma;t:\forall x::\alpha.\;\U_\ell\vdash\tau\ctor\Rightarrow A;p.B;pg'x.G
\end{matrix}}{
\begin{array}{l}
\Gamma;t:\forall x::\alpha.\;\U_\ell\vdash(\forall z::\gamma.\;t\;e)\to\tau\ctor\Rightarrow A;\ p.\;\Sigma(z::\gamma)+B;\\
\qquad pgx.\;G[\lambda b.\;g\;(\inr b)/g']\land \forall z::\gamma.\;g\;(\inl (z))\;e
\end{array}}$$
In the final rule, the notation $(z)$ where $z::\gamma$ means the tuple of elements of $z$ of type $\Sigma(z::\gamma)$: explicitly, $(z_1,\dots,z_n)=(z_1,(z_2,\dots,(x_n,()))):\Sigma(z::\gamma)$.
Note that in the base case of $\mathsf{ctor}$, we have $x=e$ where $x$ and $e$ are telescopes; this can be defined as $(x)=(e)$, or using heterogeneous equality $x_1=e_1\land x_2==e_2\land \dots\land x_n==e_n$, or using the equality recursor $\exists (h_1:x_1=e_1)\;(h_2:\rec_=\;x_2\;x_1\;h_1=e_2)\dots$. We will use $(x)=(e)$ since it is the least notationally burdensome of these options.

The final result is given by the following translation:
$$\frac{\Gamma;t:F\vdash K\spec\Rightarrow A;B;G}
{\Gamma\vdash\scott{\mu t:F.\;K}=\lambda x::\alpha.\;\{s:\W p:A.\;B\;p\mid \rec_\W\;(\lambda (p:A)\; \_.\;G\;p)\;s\;x\}}$$
In the case of a small eliminator, we just artificially lift the target universe above 1, translate it, and then propositionally truncate the resulting type and lift if back to the original universe $\ell$:
$$\frac{\Gamma;t:F\vdash K\spec\quad \neg(\Gamma;t:\forall x::\alpha.\;\U_\ell\vdash K\LE)}
{\Gamma\vdash\scott{\mu t:\forall x::\alpha.\;\U_\ell.\;K}=\lambda x::\alpha.\;\ulift^\ell\|\scott{\mu t:\forall x::\alpha.\;\U_{\ell'}.\;K}\;x\|},$$
where $\ell'$ is the maximum of $1$ and all the constructor arguments. The idea here is that since we have a small eliminator, it's impossible to tell that members of the inductive type are distinct, so we lose nothing in the propositional truncation.

\subsubsection{Translating K-like eliminators}
The hard case is when we have a K-like eliminator. In this case we must abandon $\W$-types entirely, since we have to produce a subsingleton family from the start -- propositional truncation will destroy the large elimination property, so we have to use $\acc$ instead. The zero case is easy:
$$\frac{\Gamma\vdash x::\alpha}
{\Gamma\vdash\scott{\mu t:\forall x::\alpha.\;\U_\ell.\;0}=\lambda x::\alpha.\;{\bf 0}_\ell}$$

For our purposes it will be easier to work with the following variant on $\acc$:
\begin{align*}
&\alpha:\U_\ell,\vph:\alpha\to\P,r:\alpha\to\alpha\to\P\vdash\acc^\vph_r=\mu t:\alpha\to\P.\\
&\qquad\;(\intro:\forall x:\alpha.\;\vph\;x\to(\forall y:\alpha.\;r\;y\;x\to t\;y)\to t\;x)
\end{align*}

This is just the same as $\acc_r$ but for the additional parameter $\vph$ that restricts the satisfying instances. This can be built from plain $\acc$ in our existing axiomatization as follows:

\begin{align*}
\acc^\vph_r\;x&:=\exists h:\vph\;x.\;\acc_{r'}\;(x,h)\\
\mbox{where}\qquad r'&:=\lambda x\;x':\{x:\alpha\mid \vph\;x\}.\;r\;(\pi_1\;x)\;(\pi_1\;x')
\end{align*}
Large elimination for $\acc^\vph_r$ is derivable because $\exists h:p.\;q$ has projections when $p$ is a proposition.

In the translation, we must pack up the family into a single type and then use $\acc$ for the recursive instances. Let us run an example first:
$$P=\mu t:\N\to\N\to\P.\;(\intro:\forall n:\N.\;n>2\to (\forall m.\;m<n\to t\;n\;m)\to t\;0\;n)$$
This is a large eliminating type because of the constructor's three arguments, one appears in the result ($t\;0\;n$), one is a proposition ($n>2$), and one is recursive ($\forall m.\;m<n\to t\;n\;m$).

First we pack the domain into a sigma type, in this case $\N\times\N$, and the propositional constraints go into $\vph$. The recursive arguments become the edge relation for $\acc$. Here, $(a,b)$ is accessible when there exists an $n$ such that $(a,b)=(0,n)$, $n>2$ and for all $m<n$, $(n,m)$ is accessible, so we translate this to $\vph(a,b)$ iff there exists $n$ such that $(a,b)=(0,n)$ and $n>2$, and $r\;(a',b')\;(a,b)$ iff there exists $m,n$ such that $(a,b)=(0,n)$ and $m<n$ and $(a',b')=(n,m)$.

In both clauses we introduce a variable $n$ equal to $b$ or $b'$, and this variable can be eliminated. This is true generally because of the restriction on large eliminators: every non-propositional nonrecursive argument, like $n$ here, must appear in the output type, yielding a variable-variable equality $n=b$ which can be used to eliminate $n$. However, due to potential dependencies on earlier arguments, we will delay this elimination to the recursor. So in this translation we have:
\begin{align*}
P\;x\simeq\acc_r^\vph\;(x)\qquad\mbox{where}\qquad\Gamma&\vdash \vph:=\lambda p:\Sigma(x::\alpha).\; B[p/(x)]\\
\Gamma&\vdash r:=\lambda p\;q:\Sigma(x::\alpha).\;R[p/(x')][q/(x)]\\
\Gamma,x::\alpha&\vdash B:\P\\
\Gamma,x'::\alpha,x::\alpha&\vdash R:\P
\end{align*}
where we must specify the definition of $B$ and $R$ inductively with the displayed free variables. Here the notation $B[p/(x)]$ means to replace each $x_i$ with the appropriate projection $\pi_1(\pi_2^i\;p)$ in $B$. We will also accumulate an auxiliary $\Gamma,x'::\alpha,x::\alpha\vdash S:\P$ for constructing the disjunctions in $R$.

$$\boxed{\Gamma;t:F\vdash \tau\LEctor\Rightarrow x.B;x'x.[S;R]}$$
$$\frac{}{\Gamma;t:F\vdash t\;e\LEctor\Rightarrow x.\;x=e;\ x'x.\;[x=e;\bot]}$$
$$\frac{\Gamma,t:F\vdash \beta:\U_\ell\quad \Gamma,y:\beta;t:F\vdash\tau\LEctor\Rightarrow x.B;x'x.[S;R]}
{\Gamma;t:F\vdash\forall y:\beta.\;\tau\LEctor\Rightarrow x.\exists y:\beta.\;B;x'x.[\exists y:\beta.\;S;\exists y:\beta.\;R]}$$
$$\frac{\Gamma;t:F\vdash\beta\LEctor\Rightarrow x.B;x'x.[S;R]}{\Gamma;t:F\vdash(\forall z::\gamma.\;t\;e)\to\beta\LEctor\Rightarrow x.B;x'x.[S;(S\land\exists z::\gamma.\;x'=e)\lor R]}$$
Intuitively, $S$ collects the facts that are true about the main instance argument $x$, so that in each recursive constructor we push a conjunction of $S$ with the fact $\exists z::\gamma.\;x'=e$ we need to hold for $x'$. Since we do the same thing for propositional and index arguments (just existentially generalize everything), we have collapsed both into one rule. Once we have constructed the term, we have the following rule:

$$\frac{\Gamma,t:\forall x::\alpha.\;\U_\ell\vdash \beta\LEctor\Rightarrow x.B;x'x.[S;R]}
{\Gamma\vdash\scott{\mu t:\forall x::\alpha.\;\U_\ell.\;(c:\beta)}=\lambda x::\alpha.\;\ulift^\ell(\acc^\vph_r(x))}$$
\vspace{-5mm}
\begin{align*}
\mbox{where, as before, }\vph&:=\lambda p:\Sigma(x::\alpha).\; B[p/(x)]\\
\mbox{and }r&:=\lambda p\;q:\Sigma(x::\alpha).\;R[p/(x')][q/(x)].
\end{align*}

\subsubsection{The remainder}
We have described the translation of a recursive type in great detail, but it still remains to define the introduction rules and the recursor, and show that the iota rule holds definitionally with these definitions. As these are more or less uniquely determined by the translated type of the inductive type itself, and it is yet more cumbersome than what has been thus far written, this will be left as future work, probably as part of a formalization of all of this. For now, we will proceed with the understanding that the eight inductive types $\bot,\Sigma,+,\ulift,\|\cdot\|,\W,=,\acc$ are indeed sufficient to cover all Lean-definable inductive types, and leave all this horrible induction behind.
