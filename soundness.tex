\section{Soundness}\label{sec:soundness}
\subsection{Modeling Lean in ZFC}
Fix an increasing sequence $(\kappa_n)_{n\in\N}$ of strong limit cardinals. We will say that the sequence is $n$-correct if $\kappa_0,\dots,\kappa_{n-1}$ are all inaccessible cardinals (that is, the cofinality of $\kappa_k$ is $\kappa_k$ for all $k<n$). The sequence is $\omega$-correct if it is $n$-correct for all $n$.

Note that a sequence satisfying the given properties can be proven to exist in ZFC, and a sequence which is $n$-correct can be proven to exist in ZFC${}+{}$``there are at least $n$ inaccessible cardinals'', since for any cardinal $\mu$, $\beth_\omega(\mu)$ is a strong limit cardinal greater than $\mu$.

Now let $U_0=\{\emptyset,\{\bullet\}\}$ where $\bullet=\emptyset$ (the exact identity of $\bullet$ is not important), and let $U_{n+1}=V_{\kappa_n}$.

Let $\kappa_\omega=\operatorname{sup}_{n<\omega}\kappa_n$, and $U_\omega=V_{\kappa_\omega}=\bigcup_{n<\omega} U_n$. Fix a choice function $\vep$ on $U_\omega$, that is, a function such that $\vep(x)\in x$ for all $x\in U_\omega$.

We start with the levels, which have a simple interpretation. Let $UV(\ell)$ denote the set of free universe variables in the level expression $\ell$, and similarly with $UV(e)$. (There are no universe binding operations, so all variables are free.) The expression $\scott{\ell}_v$ is defined when $v$ is a function with domain containing $FV(\ell)$ and codomain $\N$, as follows:
\begin{align*}
\scott{u}_v&=v(u)\\
\scott{0}_v&=0\\
\scott{S\ell}_v&=\scott{\ell}_v+1\\
\scott{\max(\ell,\ell')}_v&=\max(\scott{\ell}_v,\scott{\ell'}_v)\\
\scott{\imax(\ell,\ell')}_v&=\begin{cases}
0&\mbox{if }\scott{\ell'}_v=0\\
\max(\scott{\ell}_v,\scott{\ell'}_v)&\mbox{if }\scott{\ell'}_v\ne 0
\end{cases}
\end{align*}

Since $\ell\equiv\ell'$ iff $\scott{\ell}_v=\scott{\ell'}_v$ for all $v$, from unique typing we have that given $\Gamma$ and $\alpha$, there is at most one $n$ such that there exists $\ell$ with $\Gamma\vdash\alpha:\U_\ell$ and $\scott{\ell}_v=n$. We will denote this partial function, defined on expressions that are types, by $\lvl_v(\Gamma\vdash\alpha)$.
Fix a valuation $v$ on the universe variables. We define the expressions $\scott{\Gamma}_v$ when $\Gamma$ is a well formed context, and $\scott{\Gamma\vdash e}_v$ when $\Gamma\vdash e:\alpha$ for some $\alpha$, by mutual recursion on the following measure:
\begin{itemize}
\item The size of an expression $|e|$ is the sum of all its immediate subterms plus 1, except:
\begin{itemize}
\item $|\elet{x:\alpha:=e_1}{e_2}|=|e_2[e_1/x]|+1$, and
\item $|c|=|e|+1$ when $\mathsf{def}\;c:\alpha:=e$.
\end{itemize}
\item The size of a context is $|{\cdot}|=1/2$, $|\Gamma,x:\alpha|=|\Gamma|+|\alpha|$.
\item The size of an expression in context is $|\Gamma\vdash e|=|\Gamma|+|e|-1/2$.
\end{itemize}
Note that $|\Gamma|<|\Gamma\vdash e|$ and $|\Gamma\vdash\alpha|<|\Gamma,x:\alpha|$. Here $\scott{\Gamma}$ will be a set of lists of types, and $\scott{\Gamma\vdash e}$ will be a (total) function on $\scott{\Gamma}$, if it is defined at all. We will denote the evaluation at $\gamma\in\scott{\Gamma}$ by $\scott{\Gamma\vdash e}_\gamma$ (or $\scott{\Gamma\vdash e}_{v,\gamma}$ when being explicit about the universe valuation as well).

Let $\prod^n_{x\in A}B=\begin{cases}
\prod_{x\in A}B&\mbox{if }n\ne 0\\
\{\bullet\}\cap\bigcap_{x\in A}B&\mbox{if }n=0.
\end{cases}$ This is the ``modified'' product space which depends on the target universe $\ell$. Also let $[p]=\{x\in\{\bullet\}\mid p\}$, and $\bar R=\{(a,b)\mid\bullet\in R(a)(b)\}$, which translates DTT predicates and relations into their ZFC counterparts.
\begin{itemize}
\item $\scott{\cdot}=()$, the empty list. (This can be encoded as $\emptyset$.)
\item $\scott{\Gamma,x:\alpha}=\sum_{\gamma\in\scott{\Gamma}}\scott{\Gamma\vdash\alpha}_\gamma$, that is, the set of pairs $(\gamma,x)$ such that $\gamma\in\scott{\Gamma}$ and $x\in\scott{\Gamma\vdash\alpha}_\gamma$.
\item $\scott{\Gamma\vdash x}_\gamma=\pi_i(\gamma)$, where $x$ is the $i$th variable in the context.
\item $\scott{\Gamma\vdash\U_\ell}_\gamma=U_{\scott{\ell}}$
\item If $\Gamma\vdash e_1:\forall x:\alpha.\;\beta$ and $n:=\lvl(\Gamma, x:\alpha\vdash \beta)$, then $$\scott{\Gamma\vdash e_1\;e_2}_\gamma=\begin{cases}
\scott{\Gamma\vdash e_1}_\gamma(\scott{\Gamma\vdash e_2}_\gamma)&\mbox{if }n\ne 0\\
\bullet&\mbox{if }n=0.
\end{cases}$$
\item If $\Gamma,x:\alpha\vdash e:\beta:\U_\ell$, then
$$\scott{\Gamma\vdash \lambda x:\alpha.\;e}_\gamma=\begin{cases}
(x\in\scott{\Gamma\vdash \alpha}_\gamma\mapsto\scott{\Gamma,x:\alpha\vdash e}_{(\gamma,x)})&\mbox{if }\scott{\ell}\ne 0\\
\bullet&\mbox{if }\scott{\ell}=0.
\end{cases}$$
\item $\scott{\Gamma\vdash \forall x:\alpha.\;\beta}_\gamma=\prod^n_{x\in \scott{\Gamma\vdash\alpha}_\gamma}\scott{\Gamma,x:\alpha\vdash\beta}_{(\gamma,x)}$ where $n:=\lvl(\Gamma,x:\alpha\vdash\beta)$.
\item $\scott{\Gamma\vdash\elet{x:\alpha:=e_1}{e_2}}_\gamma=\scott{\Gamma\vdash e_2[e_1/x]}_\gamma$
\item $\scott{\Gamma\vdash c}_\gamma=\scott{\Gamma\vdash e}_\gamma$ when $\mathsf{def}\;c:\alpha:=e$
\item $\scott{\Gamma\vdash\bot}_\gamma=\emptyset$
\item $\scott{\Gamma\vdash\rec_\bot}_\gamma=\emptyset$ (the empty function)
\item $\scott{\Gamma\vdash\Sigma x:\alpha.\;\beta}_\gamma=\sum_{x\in \scott{\Gamma\vdash\alpha}_\gamma}\scott{\Gamma\vdash\beta}_{(\gamma,x)}$
\item $\scott{\Gamma\vdash(e_1,e_2)}_\gamma=(\scott{\Gamma\vdash e_1}_\gamma,\scott{\Gamma\vdash e_2}_\gamma)$
\item $\scott{\Gamma\vdash\pi_1\;e}_\gamma=\pi_1(\scott{\Gamma\vdash e}_\gamma)$
\item $\scott{\Gamma\vdash\pi_2\;e}_\gamma=\pi_2(\scott{\Gamma\vdash e}_\gamma)$
\item $\scott{\Gamma\vdash\alpha+\beta}_\gamma=\scott{\Gamma\vdash\alpha}_\gamma\sqcup\scott{\Gamma\vdash\beta}_\gamma$
\item $\scott{\Gamma\vdash\inl e}_\gamma=\iota_1(\scott{\Gamma\vdash\alpha}_\gamma)$
\item $\scott{\Gamma\vdash\inr e}_\gamma=\iota_2(\scott{\Gamma\vdash\beta}_\gamma)$
\item $\scott{\Gamma\vdash\rec_+\;a\;b}_\gamma$ is the function on $\scott{\Gamma\vdash\alpha}_\gamma\sqcup\scott{\Gamma\vdash\beta}_\gamma$ such that
\begin{align*}
\scott{\Gamma\vdash\rec_+\;a\;b}_\gamma(\iota_1(x))&=\scott{\Gamma\vdash a}_\gamma(x)&&\mbox{for }x\in \scott{\Gamma\vdash\alpha}_\gamma\\
\scott{\Gamma\vdash\rec_+\;a\;b}_\gamma(\iota_2(y))&=\scott{\Gamma\vdash a}_\gamma(y)&&\mbox{for }y\in \scott{\Gamma\vdash\beta}_\gamma.
\end{align*}
\item $\scott{\Gamma\vdash\ulift_\ell^{\ell'} \alpha}_\gamma=\scott{\Gamma\vdash\alpha}_\gamma$
\item $\scott{\Gamma\vdash{\uparrow}e}_\gamma=\scott{\Gamma\vdash{\downarrow}e}_\gamma=\scott{\Gamma\vdash e}_\gamma$
\item $\scott{\Gamma\vdash\|\alpha\|}_\gamma=[\scott{\Gamma\vdash\alpha}_\gamma\ne \emptyset]$
\item $\scott{\Gamma\vdash|e|}_\gamma=\bullet$
\item $\scott{\Gamma\vdash\rec_{||}^{C,\alpha}\;f}_\gamma=\bullet$
\item $\scott{\Gamma\vdash\W x:\alpha.\;\beta}_\gamma=\W_{x\in\scott{\Gamma\vdash\alpha}_\gamma}\scott{\Gamma,x:\alpha\vdash\beta}_{(\gamma,x)}$ (see below)
\item $\scott{\Gamma\vdash\sup a\;f}_\gamma=(\scott{\Gamma\vdash a}_\gamma,\scott{\Gamma\vdash f}_\gamma)$
\item $\scott{\Gamma\vdash\rec_\W\;e}_\gamma=\rec_\W(\W_{x\in\scott{\Gamma\vdash\alpha}_\gamma}\scott{\Gamma,x:\alpha\vdash\beta}_{(\gamma,x)},\scott{\Gamma\vdash e}_\gamma)$ (see below)
\item $\scott{\Gamma\vdash a=b}_\gamma=[\scott{\Gamma\vdash a}_\gamma=\scott{\Gamma\vdash b}_\gamma]$
\item $\scott{\Gamma\vdash \refl\;a}_\gamma=\bullet$
\item $\scott{\Gamma\vdash\rec_=^a\;e\;b\;h}_\gamma=\scott{\Gamma\vdash e}_\gamma$
\item $\scott{\Gamma\vdash\acc_{\alpha,r}\;x}_\gamma=[x\in\acc(\scott{\Gamma\vdash \alpha}_\gamma,\overline{\scott{\Gamma\vdash r}_\gamma})]$ (see below)
\item $\scott{\Gamma\vdash\intro_\acc\;x\;f}_\gamma=\bullet$
\item $\scott{\Gamma\vdash\rec_\acc^r\;e}_\gamma=\rec_\acc(\scott{\Gamma\vdash\alpha}_\gamma,\overline{\scott{\Gamma\vdash r}_\gamma},\scott{\Gamma\vdash e}_\gamma)$ (see below)
\item If $\Gamma\vdash\alpha:\U_\ell$ where $\scott{\ell}=0$, then
\begin{itemize}
\item $\scott{\Gamma\vdash\alpha/R}_\gamma=\scott{\Gamma\vdash\alpha}_\gamma$
\item $\scott{\Gamma\vdash\mk_R\;x}_\gamma=\bullet$
\item $\scott{\Gamma\vdash\lift^u_R\;\beta\;f\;h}_\gamma=\scott{\Gamma\vdash f}_\gamma$
\end{itemize}
Otherwise, let $\sim$ be the equivalence closure of $\overline{\scott{\Gamma\vdash R}_\gamma}$ in the following clauses:
\begin{itemize}
\item $\scott{\Gamma\vdash\alpha/R}_\gamma=\scott{\Gamma\vdash\alpha}_\gamma/{\sim}$
\item $\scott{\Gamma\vdash\mk_R\;x}_\gamma=[\scott{\Gamma\vdash x}_\gamma]_\sim$
\item $\scott{\Gamma\vdash\lift_R\;\beta\;f\;h}_\gamma$ is the function such that $\scott{\Gamma\vdash\lift_R\;\beta\;f\;h}_\gamma([x]_\sim)=\scott{\Gamma\vdash f}_\gamma(x)$
\end{itemize}
\item $\scott{\Gamma\vdash\sound_R}_\gamma=\bullet$
\item $\scott{\Gamma\vdash\mathsf{propext}}_\gamma=\bullet$
\item $\scott{\Gamma\vdash\mathsf{choice}\;\alpha\;h}_\gamma=\vep(\scott{\Gamma\vdash\alpha}_\gamma)$
\end{itemize}

\subsubsection{Definition of $\W$-types in ZFC}
If $A$ is a set and $B(x)$ is a family of sets indexed by $x\in A$, then $\W_{x\in A}B(x)$ is a set, defined as the intersection of all sets $W$ such that $(a,f)\in W$ whenever $a\in A$ and $f:B(a)\to W$. If $\mathrm{cf}(\lambda)>\mathrm{sup}_{x\in A}|B(x)|$, then $V_\lambda$ is an upper bound for $W$, since $\rank\circ f$ is a sequence of ordinals of length $B(a)<\mathrm{cf}(\lambda)$. Thus, $U_{n+1}$ is closed under $\W$-types if the $\kappa$ sequence is $(n+1)$-correct.

The recursor $F:=\rec_{\W}(W,e):W\to V$ is defined by transfinite recursion on $x\in W$: Assuming that $F(y)$ is defined for all $y$ with $\rank y<\rank x$, we let $F(a,f)=e(a)(f)(F\circ f)$ when $x=(a,f)$ is a pair. Note that $\rank f(y)<\rank f<\rank x$ for all $y\in\mathrm{dom}\;f$, so the function is well-defined.

\subsubsection{Definition of $\acc$ in ZFC}
If $R\subseteq A\times A$ is a relation, $\acc(A,R)\subseteq A$ is the set of all elements $x\in A$ that are $R$-accessible, or equivalently, $R$ restricted to the set $\{y\mid (y,x)\in R^*\}$ is well-founded. It is the smallest set $C$ such that if $y\in C$ for all $y$ such that $(y,x)\in R$, then $x\in C$.

Note that $\acc(A,R)$ is itself well-founded by $R$, because any descending sequence in $\acc(A,R)$ must begin at some point of it, and can only continue finitely long from there. We define the recursor as $\rec_\acc(A,R,e)=(x\in A\mapsto (h\in [x\in\acc(A,R)]\mapsto F(x))$, where $F:\acc(A,R)\to V$ is defined by recursion on $\acc(A,R)$ along $R$, such that $F(x)=e(x)(\bullet)(y\in A\mapsto (h\in[(y,x)\in R]\mapsto F(y)))$.

\begin{remark}\label{rem:inacc_dprod}
If $\lambda$ is an inaccessible cardinal, and $A\in V_\lambda$, $B:A\to V_\lambda$, then $\prod_{x\in A} B(x)\in V_\lambda$, because every element of $V_\lambda$ has cardinality $<\lambda$, so $B$ is bounded in rank (by some $\mu<\lambda$), and then the rank of $\prod_{x\in A} B(x)$ is at most $\mu+4$ or so.
\end{remark}

\subsection{Soundness}
Fix a universe valuation $v$ for this section.
\begin{lemma}[Basics]
\begin{itemize}
\item (Weakening) If $\Gamma\vdash e:\alpha$ and $\vdash\Gamma,\Delta\ok$, and $(\gamma,\delta)\in\scott{\Gamma,\Delta}$, then $\scott{\Gamma,\Delta\vdash e}_{\gamma,\delta}=\scott{\Gamma\vdash e}_\gamma$.
\item (Substitution) If $\Gamma,x:\alpha\vdash e_1:\beta$, $\Gamma\vdash e_2:\alpha$, $\gamma\in\scott{\Gamma}$, and $z:=\scott{\Gamma\vdash e_2}_\gamma\in\scott{\Gamma\vdash\alpha}_\gamma$, then $\scott{\Gamma\vdash e_1[e_2/x]}_\gamma=\scott{\Gamma,x:\alpha\vdash e_1}_{(\gamma,z)}$.
\end{itemize}
\end{lemma}
\begin{proof}
Straightforward. (In the substitution lemma, we are assuming soundness for $e_2$, because we haven't proven it yet.)
\end{proof}

\begin{theorem}[Soundness]\label{thm:sound}
\begin{itemize}
\item If $\Gamma\vdash \alpha:\U_\ell$ where $\scott{\ell}=0$, then $\scott{\Gamma\vdash\alpha}_\gamma\subseteq\{\bullet\}$.
\item If $\Gamma\vdash e:\alpha$ where $\lvl(\Gamma\vdash\alpha)=0$, then $\scott{\Gamma\vdash e}_\gamma=\bullet$.
\item If $\Gamma\vdash e:\alpha$, then there exists an $k$ such that if the $\kappa$ sequence is $k$-correct, then for all $\gamma\in\scott{\Gamma}$, $\scott{\Gamma\vdash e}_{v,\gamma}\in \scott{\Gamma\vdash \alpha}_{v,\gamma}$.
\item If $\Gamma\vdash e\equiv e'$, then there exists an $k$ such that if the $\kappa$ sequence is $k$-correct, then for all $\gamma\in\scott{\Gamma}$, $\scott{\Gamma\vdash e}_{v,\gamma}=\scott{\Gamma\vdash e'}_{v,\gamma}$.
\end{itemize}
\end{theorem}
\begin{proof}
The proof is constructive for the value of $k$; it is essentially just the max of all universe numbers that appear in the course of the proof. We will not spend much time discussing it, but it is worth noting that we may have $\Gamma\vdash e:\alpha$ where $\scott{\alpha}$ is not a member of the expected universe, without assuming a higher value of $k$ than the one that appears in the proof.

(There is nothing surprising in this proof, except perhaps the fact that I took the trouble to write it down.)

Part 1 is a special case of part 3, but does not require the $k$ assumption. We will prove it in parallel with the other parts.

For brevity of notation, we will adopt the convention that $\bar\alpha$ means $\scott{\Gamma\vdash\alpha}_\gamma$, $\bar\beta(x)$ means $\scott{\Gamma,x:\alpha\vdash\beta}_{(\gamma,x)}$, and so on, where $\Gamma$ and $\gamma$ are understood from context.
\begin{itemize}
\item Weakening. We have $\scott{\Gamma\vdash e}_\gamma\in\scott{\Gamma\vdash\beta}_\gamma$ by the IH, and $\scott{\Gamma,x:\alpha\vdash e}_{(\gamma,x)}=\scott{\Gamma\vdash e}_\gamma$ and $\scott{\Gamma,x:\alpha\vdash\beta}_{(\gamma,x)}=\scott{\Gamma\vdash\beta}_\gamma$ by the weakening lemma, so $\scott{\Gamma,x:\alpha\vdash e}_{(\gamma,x)}\in \scott{\Gamma,x:\alpha\vdash\beta}_{(\gamma,x)}$.
\item Conversion. $\Gamma\vdash e:\alpha$ and $\Gamma\vdash\alpha\equiv\beta$. Then by the IH $\bar e\in\bar \alpha=\bar\beta$. Parts 1 and 2 follow from the first IH, since $\lvl(\Gamma\vdash\alpha)=\lvl(\Gamma\vdash\beta)$.
\item Variable. $\scott{\Gamma,x:\alpha\vdash x}_{(\gamma,x)}=x$, so $(\gamma,x)\in\scott{\Gamma,x:\alpha}$ implies $x\in\scott{\Gamma\vdash\alpha}_\gamma=\scott{\Gamma,x:\alpha\vdash\alpha}_{(\gamma,x)}$ by the weakening lemma.
\item Universe. $\scott{\vdash\U_{\ell}}_{()}=U_{\scott{\ell}}\in U_{\scott{\ell}+1}=\scott{\vdash\U_{S\ell}}_{()}$ since the $U$ universes form a membership hierarchy. Parts 1 and 2 do not apply since $\U_{S\ell}\not\equiv\P$.
\item Application. Suppose $\Gamma\vdash e_1:\forall x:\alpha.\;\beta$ and $\Gamma\vdash e_2:\alpha$. Then $\Gamma,x:\alpha\vdash\beta\type$, so $n:=\lvl(\Gamma,x:\alpha\vdash\beta)$ is defined. By the IH, $\bar{e_1}\in\prod^n_{x\in \bar\alpha}\bar\beta(x)$ and $\bar{e_2}_\gamma\in\bar\alpha$. The result type is $\lvl(\Gamma\vdash\beta[e_2/x])=n$, so part 2 only applies when $n=0$.
\begin{itemize}
\item If $n=0$, then $\bar{e_1}=\bullet\in\bigcap_{x\in \bar\alpha}\bar\beta(x)$, so in particular $\scott{\Gamma\vdash e_1\;e_2}_\gamma=\bullet\in\bar\beta(\bar{e_2})=\scott{\Gamma\vdash\beta[e_2/x]}_\gamma$ by the substitution lemma.
\item If $n\ne 0$, then $\bar{e_1}\in\prod_{x\in \bar\alpha}\bar\beta(x)$, so $\scott{\Gamma\vdash e_1\;e_2}_\gamma=\bar{e_1}(\bar{e_2})\in\bar\beta(\bar{e_2})=\scott{\Gamma\vdash\beta[e_2/x]}_\gamma$ by the substitution lemma.
\end{itemize}
\item Lambda. Suppose $\Gamma,x:\alpha\vdash e:\beta$. Then $n:=\lvl(\Gamma,x:\alpha\vdash\beta)$ is defined. By the IH, $\bar e(x)\in\bar\beta(x)$ for all $x\in\bar\alpha$. The result type is $\lvl(\Gamma\vdash\forall x:\alpha.\;\beta)=\imax(\lvl(\Gamma\vdash\alpha),n)$ which is $0$ iff $n$ is, so part 2 only applies when $n=0$.
\begin{itemize}
\item If $n=0$, then $\bar e(x)=\bullet\in\bar\beta(x)$ for all $x\in\bar\alpha$, so $\scott{\Gamma\vdash\lambda x:\alpha.\;e}_\gamma=\bullet\in\bigcap_{x\in \bar\alpha}\bar\beta(x)$.
\item If $n\ne 0$, then $\scott{\Gamma\vdash\lambda x:\alpha.\;e}_\gamma=(x\in\bar\alpha\mapsto\bar e(x))\in\prod_{x\in \bar\alpha}\bar\beta(x)$.
\end{itemize}
\item Forall. Suppose $\Gamma\vdash\alpha:\U_{\ell_1}$ and $\Gamma,x:\alpha\vdash\beta:\U_{\ell_2}$. Then $n:=\lvl(\Gamma,x:\alpha\vdash\beta)=\scott{\ell_2}$. By the IH, $\bar\alpha\in U_{\scott{\ell_1}}$ and $\bar\beta(x)\in U_n$ for all $x\in\bar\alpha$. The result type is a universe so part 2 does not apply, and part 1 applies only when $n=0$.
\begin{itemize}
\item If $n=0$, then $\scott{\Gamma\vdash\forall x:\alpha.\;\beta}_\gamma=\{\bullet\}\cap\bigcap_{x\in \bar\alpha}\bar\beta(x)\subseteq\{\bullet\}$.
\item If $n\ne 0$, then $\scott{\Gamma\vdash\forall x:\alpha.\;\beta}_\gamma=\prod_{x\in \bar\alpha}\bar\beta(x)\in U_{\max(\scott{\ell_1},\scott{\ell_2})}$, provided that the $\kappa$ sequence is $\max(\scott{\ell_1},\scott{\ell_2})$-correct, because if $\kappa_{\max(\scott{\ell_1},\scott{\ell_2})-1}$ is inaccessible then\\ $U_{\max(\scott{\ell_1},\scott{\ell_2})}$ is closed under dependent products.
\end{itemize}
\item $\bot$: $\scott{\vdash\bot}_{()}=\emptyset\subseteq\{\bullet\}$.
\item $\rec_\bot$: $\scott{\Gamma\vdash\rec^C_\bot}_\gamma=\emptyset\in\scott{\Gamma\vdash\bot\to C}_\gamma=\prod_{x\in\emptyset}\bar C$.
\item $\Sigma$: Assuming the $\kappa$ sequence is $k$-correct where $k=\max(1,\scott{\ell_1},\scott{\ell_2})$, if $\bar\alpha\in U_{\scott{\ell_1}}\subseteq U_k$ and $\bar\beta(x)\in U_{\scott{\ell_2}}\subseteq U_k$ for all $x\in \bar\alpha$, the family is bounded, so $\sum_{x\in\bar\alpha}\bar\beta(x)\in U_k$.
\item Pair: If $\bar{e_1}\in\bar\alpha$ and $\bar{e_2}\in \scott{\Gamma\vdash\beta[e_1/x]}_\gamma=\bar\beta(\bar{e_1})$, then $\scott{\Gamma\vdash(e_1,e_2)}_\gamma=(\bar{e_1},\bar{e_2})\in \sum_{x\in\bar\alpha}\bar\beta(x)$.
\item $\pi_1$: If $\bar e\in \sum_{x\in\bar\alpha}\bar\beta(x)$, then $\scott{\Gamma\vdash\pi_1\;e}_\gamma=\pi_1(\bar e)\in\bar\alpha$.
\item $\pi_2$: If $\bar e\in \sum_{x\in\bar\alpha}\bar\beta(x)$, then $\scott{\Gamma\vdash\pi_2\;e}_\gamma=\pi_2(\bar e)\in\bar\beta(\pi_1(\bar e))=\bar\beta(\scott{\Gamma\vdash\pi_1\;e}_\gamma)=\scott{\Gamma\vdash\beta[\pi_1\;e/x]}_\gamma$.
\item $+$: If $k:=\max(1,\scott{\ell_1},\scott{\ell_2})$, and $\bar\alpha\in U_{\scott{\ell_1}}\subseteq U_k$ and $\bar\beta\in U_{\scott{\ell_2}}\subseteq U_k$, then $\scott{\Gamma\vdash\alpha+\beta}_\gamma=\bar\alpha\sqcup\bar\beta\in U_k$, because $\rank(\bar\alpha\sqcup \bar\beta)\le\max(\rank \bar\alpha,\rank \bar\beta)+2$ (when encoded as marked pairs), so $V_\lambda$ is closed under disjoint unions whenever $\lambda$ is a limit ordinal.
\item $\mathsf{inl}$: If $\bar e\in\bar\alpha$, then $\scott{\Gamma\vdash\inl e}_\gamma=\iota_1(\bar e)\in\bar\alpha\sqcup\bar\beta=\scott{\Gamma\vdash\alpha+\beta}_\gamma$. (We don't need the second IH.) $\mathsf{inr}$ is similar.
\item $\rec_+$: By the IH, $\bar C:\bar\alpha\sqcup\bar\beta\to U_{\scott{\ell}}$.

$\scott{\Gamma\vdash\rec^C_\bot\;a\;b}_\gamma\in\prod_{x\in\bar\alpha\sqcup\bar\beta}\bar C(x)$ because it was defined as a function such that $\scott{\Gamma\vdash\rec^C_\bot\;a\;b}_\gamma(\iota_1(x))=\bar a(x)\in\bar C(\iota_1(x))$, and $\scott{\Gamma\vdash\rec^C_\bot\;a\;b}_\gamma(\iota_2(x))=\bar b(x)\in\bar C(\iota_2(x))$.
\item $\ulift$: If $\bar\alpha\in U_{\scott{\ell}}$ and $\ell\le\ell'$, then $\scott{\ell}\le\scott{\ell'}$ so $\scott{\Gamma\vdash\ulift_{\ell}^{\ell'}\;\alpha}_\gamma=\bar\alpha\in U_{\scott{\ell}}\subseteq U_{\scott{\ell'}}$.
\item $\uparrow$ and $\downarrow$ are trivial from the IH.
\item $\|\cdot\|$: $\scott{\Gamma\vdash\|\alpha\|}_\gamma=[\bar\alpha\ne\emptyset]\subseteq\{\bullet\}$ (we don't need the IH).
\item $|\cdot|$: If $\bar e\in\bar\alpha$, then $\bar\alpha\ne\emptyset$ so $\scott{\Gamma\vdash|e|}_\gamma=\bullet\in[\bar\alpha\ne\emptyset]=\scott{\Gamma\vdash\|\alpha\|}_\gamma$.
\item $\rec_{||}$: To show $\scott{\Gamma\vdash\rec_{||}\;f}_\gamma=\bullet\in\scott{\Gamma\vdash\|\alpha\|\to C}_\gamma$, it suffices to show that if $x\in[\bar\alpha\ne\emptyset]$ (i.e. $\bar\alpha\ne\emptyset$), then $\bullet\in\bar C$. Let $y\in\bar\alpha$. Then $\bar f=\bullet\in\bigcap_{x\in\bar\alpha}\bar C$, so $\bullet\in\bar C$, using $x:=y$.
\item $\W$: Similar to the $\Sigma$ case, assuming the $\kappa$ sequence is $k$-correct where $k=\max(1,\scott{\ell_1},\scott{\ell_2})$, since we have already observed that $V_\lambda$ where $\lambda$ is inaccessible is closed under $\W$-types.
\item $\sup$: $\scott{\Gamma\vdash \sup a\;f}_\gamma=(\bar a,\bar f)\in\W_{x\in\bar\alpha}\bar\beta(x)$ since $\bar a\in\bar\alpha$ and $\bar f:\bar\beta(\bar a)\to\W_{x\in\bar\alpha}\bar\beta(x)$. (Application of the definition, IH, and substitution theorem.)
\item $\rec_\W$: Let $W:=\scott{\Gamma\vdash\W x:\alpha.\;\beta}_\gamma=\W_{x\in\bar\alpha}\bar\beta(x)$. By the IH and applying the definitions, $$\bar e:=\bar e\in\prod_{a\in\bar\alpha}\prod_{f:\bar\beta(a)\to W}\Big[\prod_{b\in\bar\beta(a)}\bar C(f(b))\Big]\to \bar C(a,f).$$
Now $\scott{\Gamma\vdash \rec_\W\;e}_\gamma=\rec_\W(W,\bar e)=:F$ is defined as a function on $W$, so it suffices to check that $F(w)\in\bar C(w)$ for all $w\in W$. By induction on $w\in W$, it suffices to check that $F(a,f)\in\bar C(a,f)$ when $a\in\bar\alpha$ and $f:\bar\beta(a)\to W$ satisfies $F(f(b))\in\bar C(f(b))$ for all $b\in\bar\beta(a)$. Since $F(a,f)=\bar e(a)(f)(F\circ f)$, and $F\circ f\in \prod_{b\in\bar\beta(a)}\bar C(f(b))$ because $F(f(b))\in\bar C(f(b))$ by assumption, we have $\bar e(a)(f)(F\circ f)\in\bar C(a,f)$ as desired.
\item Equality: $\scott{\Gamma\vdash a=b}_\gamma=[\bar a=\bar b]\subseteq\{\bullet\}$ (we don't need the IH).
\item $\refl$: $\scott{\Gamma\vdash \refl\;a}_\gamma=\bullet\in\scott{\Gamma\vdash a=a}_\gamma=[\bar a=\bar a]$ because $\bar a=\bar a$ (we don't need the IH).
\item $\rec_=$: We have $\bar a\in\bar\alpha$, $\bar C:\bar\alpha\to U_{\scott{\ell}}$, and $\bar e\in\bar C(\bar a)$ from the IH. To show $\scott{\rec_=\;e}_\gamma\in\scott{\forall b:\alpha.\;a=b\to C\;b}_\gamma=\prod_{b\in\bar\alpha}\prod_{h\in[\bar a=b]}\bar C(b)$, suppose $b\in\bar\alpha$ and $h\in[\bar a=b]$. Then $\bar a=b$, so $\scott{\rec_=\;e}_\gamma(b)(h)=\bar e\in \bar C(\bar a)=\bar C(b)$.
\item $\acc$: If $\bar r:\bar\alpha\to\bar\alpha\to U_0$ and $\bar x\in\bar\alpha$, then $\scott{\Gamma\vdash\acc_r\;x}_\gamma=[x\in\acc(\bar\alpha,\overline{\bar r})]\subseteq\{\bullet\}$.
\item $\intro_\acc$: If $\bar x:\bar\alpha$ and $\bar f=\bullet\in\bigcap_{y\in\bar\alpha}[\bullet\in (y,\bar x)\in\overline{\bar r}]\to[y\in\acc(\bar\alpha,\overline{\bar r})]$ (applying the definitions and IH), then for all $y\in\bar\alpha$ with $(y,\bar x)\in\overline{\bar r}$, $y\in\acc(\bar\alpha,\overline{\bar r})$, so $\bar x$ is $\overline{\bar r}$-accessible, i.e. $\bar x\in\acc(\bar\alpha,\overline{\bar r})$.

Thus, $\scott{\Gamma\vdash\intro_\acc\;f\;x}_\gamma=\bullet\in\scott{\Gamma\vdash\acc_r\;x}_\gamma=[\bar x\in\acc(\bar\alpha,\overline{\bar r})]$.
\item $\rec_\acc$: We have $\bar C:\bar\alpha\to U_{\scott{\ell}}$, and
$$\bar e\in\prod_{x\in\bar\alpha}[\forall y\in\bar\alpha,(y,x)\in\overline{\bar r}\to y\in\acc(\bar\alpha,\overline{\bar r})]\to\Big(\prod_{y\in\bar\alpha}[(y,x)\in\overline{\bar r}]\to\bar C(y)\Big)\to\bar C(x).$$
We want to show $\scott{\rec_\acc\;e}_\gamma\in\prod_{x\in\bar\alpha}[x\in\acc(\bar\alpha,\overline{\bar r})]\to\bar C(x)$. It was defined as a function on this domain, so let $x\in\bar\alpha$ and $h\in[x\in\acc(\bar\alpha,\overline{\bar r})]$; then $x\in\acc(\bar\alpha,\overline{\bar r})$ and $h=\bullet$, and $\scott{\rec_\acc\;e}_\gamma(x)(h)=F(x)$ for the function $F$ in the definition of $\rec_\acc$. We prove that $F(x)\in\bar C(x)$ by induction on $x\in\acc(\bar\alpha,\overline{\bar r})$.

Suppose that for all $y\in\acc(\bar\alpha,\overline{\bar r})$, if $(y,x)\in\overline{\bar r}$ then $F(y)\in\bar C(y)$. Then $F(x)=\bar e(x)(\bullet)(y\in \bar\alpha\mapsto(h\in[(y,x)\in\overline{\bar r}]\mapsto F(y)))$. Clearly $x\in\bar\alpha$, and $\bullet\in[(y,x)\in\overline{\bar r}]$. Also, $(y\in \bar\alpha\mapsto(h\in[(y,x)\in\overline{\bar r}]\mapsto F(y)))\in\prod_{y\in\bar\alpha}[(y,x)\in\overline{\bar r}]\to\bar C(y)$ because if $y\in\bar\alpha$ and $h\in[(y,x)\in\overline{\bar r}]$, then $(y,x)\in\overline{\bar r}$ so $F(y)\in\bar C(y)$ by the IH. Thus $F(x)\in\bar C(x)$.
\item Quotients. Suppose $\Gamma\vdash\alpha:\U_\ell$, $n:=\scott{\ell}$, and $\Gamma\vdash r:\alpha\to\alpha\to\P$. Let $\sim$ be the equivalence closure of $\overline{\bar r}$. (We will assume this for the next few cases to do with quotients.)
\begin{itemize}
\item If $n=0$, then $\scott{\Gamma\vdash \alpha/r}_\gamma=\bar\alpha\subseteq\{\bullet\}$.
\item If $n\ne 0$, then $\scott{\Gamma\vdash \alpha/r}_\gamma=\bar\alpha/{\sim}\in U_n$ because $\bar\alpha/{\sim}$ is contained in the double powerset of $\bar\alpha\in\U_n$.
\end{itemize}
\item $\mk$: Suppose additionally $\Gamma\vdash x:\alpha$, so $\bar x\in\bar \alpha$ from the IH.
\begin{itemize}
\item If $n=0$, then $\scott{\Gamma\vdash\mk_r\;x}_\gamma=\bullet=\bar x\in\bar\alpha=\scott{\Gamma\vdash \alpha/r}_\gamma$.
\item If $n\ne 0$, then $\scott{\Gamma\vdash\mk_r\;x}_\gamma=[\bar x]_{\sim}\in\bar\alpha/{\sim}=\scott{\Gamma\vdash \alpha/r}_\gamma$.
\end{itemize}
\item $\lift$: Suppose $\Gamma\vdash\beta:\U_{\ell'}$ with $\scott{\ell'}\ge 1$, and $\Gamma\vdash f:\alpha\to\beta$, and\\
$\Gamma\vdash h:\forall x\;y:\alpha.\;R\;x\;y\to f\;x=f\;y$.
\begin{itemize}
\item If $n=0$, then $\scott{\Gamma\vdash\lift_r\;\beta\;f\;h}_\gamma=\bar f\in\scott{\Gamma\vdash\alpha/r\to\beta}_\gamma=\prod_{x\in\bar\alpha}\bar\beta$ by the IH.
\item If $n\ne 0$, then from the IH $\bar f:\bar\alpha\to\bar\beta$ and
$$\bar h\in\bigcap_{x\in\bar\alpha}\bigcap_{y\in\bar\alpha}[(x,y)\in\overline{\bar r}\to\bar f(x)=\bar f(y)].$$
Therefore $\forall x,y\in \bar\alpha.\;(x,y)\in\overline{\bar r}\to\bar f(x)=\bar f(y)$, so since the property $\bar f(x)=\bar f(y)$ is an equivalence relation that contains $\overline{\bar r}$, we have $x\sim y\to \bar f(x)=\bar f(y)$, so there is a well defined function $F:\bar\alpha/{\sim}\to\bar\beta$ such that $F([x]_{\sim})=\bar f(x)$, and $\scott{\Gamma\vdash\lift_r\;\beta\;f\;h}_\gamma$ was defined to be this function. Thus $\scott{\Gamma\vdash\lift_r\;\beta\;f\;h}_\gamma\in \scott{\Gamma\vdash\alpha/r\to\beta}_\gamma$.
\end{itemize}
\item $\sound$: We want to verify that $\scott{\Gamma\vdash\sound_r}_\gamma=\bullet\in\scott{\Gamma\vdash\forall x\;y:\alpha.\;r\;x\;y\to\mk_r\;x=\mk_r\;y}_\gamma$, or after expansion,
$$\bullet\in\bigcap_{x\in\bar\alpha}\bigcap_{y\in\bar\alpha}[(x,y)\in\overline{\bar r}\to[x]_{\sim}=[y]_{\sim}].$$
Let $x,y\in\bar\alpha$, and suppose $(x,y)\in\overline{\bar r}$; then since $\sim$ contains $\overline{\bar r}$, $x\sim y$ and hence $[x]_{\sim}=[y]_{\sim}$.
\item $\mathsf{propext}$: We want to verify that $\scott{\vdash\mathsf{propext}}_{()}=\bullet\in\scott{\vdash\forall p\;q:\P.\;(p\leftrightarrow q)\to p=q}_{()}$. Suppose $p,q\in U_0$. Then $p,q\subseteq\{\bullet\}$. If $\bullet\in p\leftrightarrow\bullet\in q$, then either $\bullet\in p$ and $\bullet\in q$, so $p=\{\bullet\}=q$, or $\bullet\notin p,q$, so $p=\emptyset=q$.
\item $\mathsf{choice}$: Let $\Gamma\vdash\alpha:\U_\ell$ and $\Gamma\vdash h:\|\alpha\|$. Then $\bar h\in[\bar\alpha\ne\emptyset]$, so $\bar\alpha\ne\emptyset$, and $\bar\alpha\in U_{\scott{\ell}}\subseteq U_\omega$, so since $\vep$ is a choice function on $U_\omega$, $\scott{\Gamma\vdash\mathsf{choice}\;\alpha\;h}_\gamma=\vep(\bar\alpha)\in\bar\alpha$.
\end{itemize}
This completes the proof of parts 1-3; now we consider the equivalence rules, which only involves part 4.
\begin{itemize}
\item Reflexivity, symmetry and transitivity follow since $\bar e=\bar e'$ is an equivalence relation.
\item Compatibility. This expresses the fact that each syntax constructor such as $\scott{\Gamma\vdash \alpha+\beta}_\gamma$ is defined only in terms of $\scott{\Gamma\vdash \alpha}_\gamma$ and $\scott{\Gamma\vdash\beta}_\gamma$. When a case split on $\scott{\ell}=0$ is done, by unique typing it must be the same for both sides (since $e$ and $e'$ have the same type).
\item Beta. Suppose $\Gamma,x:\alpha\vdash e:\beta$ and $\Gamma\vdash e':\alpha$, so that by the inductive hypothesis $\bar e(x)\in\bar\beta(x)$ for all $x\in\bar\alpha$, and $\bar e'\in\bar\alpha$. Let $n=\lvl(\Gamma,x:\alpha\vdash\beta)$.
\begin{itemize}
\item If $n=0$, then $\scott{\Gamma\vdash(\lambda x:\alpha.\;e)\;e'}_\gamma=\bullet=\scott{\Gamma\vdash e[e'/x]}_\gamma$ because $e[e'/x]$ is a proof (by part 2).
\item If $n\ne 0$, then $\scott{\Gamma\vdash(\lambda x:\alpha.\;e)\;e'}_\gamma=(x\in\bar\alpha\mapsto\bar e(x))(\bar e')=\bar e(\bar e')=\scott{\Gamma\vdash e[e'/x]}_\gamma$ by the substitution lemma.
\end{itemize}
\item Eta. Suppose $\Gamma\vdash e:\forall y:\alpha.\;\beta$, so that by the inductive hypothesis $\bar e\in\prod^n_{y\in\bar\alpha}\bar\beta(y)$ where $n:=\lvl(\Gamma,x:\alpha\vdash\beta)$.
\begin{itemize}
\item If $n=0$, then $\scott{\Gamma\vdash \lambda x:\alpha.\;e\;x}_\gamma=\bullet=\scott{\Gamma\vdash e}_\gamma$ because $e$ is a proof.
\item If $n\ne 0$, then $\scott{\Gamma\vdash \lambda x:\alpha.\;e\;x}_\gamma=(x\in\bar\alpha\mapsto\bar e(x))=\bar e$ by function extensionality in ZFC.
\end{itemize}
\item Proof irrelevance. If $\Gamma\vdash h,h':p:\P$, then by part 2 of the theorem, $\scott{\Gamma\vdash h}_\gamma=\bullet=\scott{\Gamma\vdash h'}_\gamma$.
\item Delta. If $\mathsf{def}\;c:\alpha:=e$, then $\scott{\Gamma\vdash c}_\gamma=\scott{\Gamma\vdash e}_\gamma$ by definition.
\item Zeta. If $\mathsf{def}\;c:\alpha:=e$, then $\scott{\Gamma\vdash\elet{x:\alpha:=e_1}{e_2}}_\gamma=\scott{\Gamma\vdash e_2[e_1/x]}_\gamma$ by definition. (We don't use the substitution lemma here because it is not necessarily true that $\Gamma,x:\alpha\vdash e_2$ is well typed.)
\item Quotient iota. $\scott{\Gamma\vdash \lift_r\;\beta\;f\;h\;(\mk_r\;a)}_\gamma=\scott{\Gamma\vdash \lift_r\;\beta\;f\;h}_\gamma([\bar a]_{\sim})=\bar f(\bar a)$ by definition (we showed it is well defined given the assumptions on $\alpha,r,\beta,f,h$ already).
\item $\pi_1$ iota. $\scott{\Gamma\vdash\pi_1(a,b)}_\gamma=\pi_1(\bar a,\bar b)=\bar a$.
\item $\pi_2$ iota. $\scott{\Gamma\vdash\pi_2(a,b)}_\gamma=\pi_2(\bar a,\bar b)=\bar b$.
\item $\inl$ iota. $\scott{\Gamma\vdash\rec_+\;a\;b\;(\inl x)}_\gamma=\scott{\Gamma\vdash\rec_+\;a\;b}_\gamma(\iota_1(\bar x))=\bar a(\bar x)=\scott{\Gamma\vdash a\;x}_\gamma$.
\item $\inr$ iota. $\scott{\Gamma\vdash\rec_+\;a\;b\;(\inr x)}_\gamma=\scott{\Gamma\vdash\rec_+\;a\;b}_\gamma(\iota_2(\bar x))=\bar b(\bar x)=\scott{\Gamma\vdash b\;x}_\gamma$.
\item $\ulift$ iota. $\scott{\Gamma\vdash{\downarrow\uparrow}x}_\gamma=\scott{\Gamma\vdash x}_\gamma$ by definition.
\item $\W$ iota. Letting $F:=\rec_\W(\W_{x\in\bar\alpha}\bar\beta(x),\bar e)$, we have $\scott{\Gamma\vdash\rec_\W\;e\;(\sup\;a\;f)}_\gamma=F(\bar a,\bar f)=\bar e(\bar a)(\bar f)(F\circ \bar f)$ on the one hand, and $\scott{\Gamma\vdash e\;a\;f\;(\lambda b:\beta[a/x].\;\rec_\W\;e\;(f\;b))}_\gamma=\bar e(\bar a)(\bar f)(b\in\bar\beta(\bar a)\mapsto F(f(b)))$ on the other; and $F\circ \bar f=(b\in\bar\beta(\bar a)\mapsto F(f(b)))$ because $\bar\beta(\bar a)$ is the domain of $f$.
\item $=$ iota. $\scott{\Gamma\vdash\rec_=\;e\;a\;h}_\gamma=\bar e$ by definition.
\item $\acc$ iota. If $F:\acc(\bar\alpha,\overline{\bar r})\to V$ is the function defined in $\rec_\acc(\bar\alpha,\overline{\bar r},\bar e)$, then we have
\begin{align*}
\scott{\Gamma\vdash\rec_\acc\;e\;x\;(\intro_\acc\;x\;f)}_\gamma&=\rec_\acc(\bar\alpha,\overline{\bar r},\bar e)(x)(\bullet)=F(x)\\
&=e(\bar x)(\bullet)(y\in\bar\alpha\mapsto(h\in[(y,x)\in\overline{\bar r}]\mapsto F(y))\\
&=e(\bar x)(\bar f)(y\in\bar\alpha\mapsto(h\in[(y,x)\in\overline{\bar r}]\mapsto F(y))\\
&=\scott{\Gamma\vdash e\;x\;f\;(\lambda (y:\alpha)\;(h:r\;y\;x).\;\rec_\acc\;e\;y\;(f\;y\;h))}_\gamma
\end{align*}
where $\bar f=\bullet$ because $f$ is a proof.
\item $\ulift$ eta. $\scott{\Gamma\vdash{\uparrow\downarrow}x}_\gamma=\scott{\Gamma\vdash x}_\gamma$ by definition.
\item $\Sigma$ eta. If $\Gamma\vdash p:\Sigma x:\alpha.\;\beta$, then $\bar p\in\sum_{x\in\bar\alpha}\bar\beta(x)$, so $\bar p=(x,y)$ is a pair, so  $\scott{\Gamma\vdash(\pi_1\;p,\pi_2\;p)}_\gamma=(\pi_1(\bar p),\pi_2(\bar p))=(x,y)=\bar p$.
\end{itemize}
\end{proof}

\begin{corollary}
Lean is consistent if $\mathrm{ZFC}+\{\mbox{there are n inaccessible cardinals}\mid n\in\omega\}$ is. That is, there is no proof of $\bot$ that is verified by the Lean kernel.
\end{corollary}
\begin{proof}
Suppose $\Vdash e:\bot$ (the algorithmic typing judgment). Then $\vdash e:\bot$ since algorithmic equality implies definitional equality. Let $v$ be the universe valuation that sets every variable to 0, and let $(\kappa_i)_{i\in\omega}$ be a cardinal sequence which is $n$-correct with $n$ sufficiently large to satisfy the assumption of \autoref{thm:sound}.  Then $\scott{\vdash e}_{()}\in\scott{\vdash\bot}_{()}=\emptyset$, a contradiction.
\end{proof}